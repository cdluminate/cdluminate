%% LyX 2.3.3 created this file.  For more info, see http://www.lyx.org/.
%% Do not edit unless you really know what you are doing.
\documentclass[twocolumn,english]{article}
\usepackage{mathptmx}
\renewcommand{\ttdefault}{cmtt}
\usepackage[T1]{fontenc}
\usepackage[latin9]{inputenc}
\usepackage{geometry}
\geometry{verbose,tmargin=2.5cm,bmargin=2.5cm,lmargin=2cm,rmargin=2cm}
\setcounter{tocdepth}{2}
\usepackage{color}
\usepackage{babel}
\usepackage[unicode=true,pdfusetitle,
 bookmarks=true,bookmarksnumbered=false,bookmarksopen=false,
 breaklinks=false,pdfborder={0 0 1},backref=false,colorlinks=true]
 {hyperref}

\makeatletter

%%%%%%%%%%%%%%%%%%%%%%%%%%%%%% LyX specific LaTeX commands.
%% Because html converters don't know tabularnewline
\providecommand{\tabularnewline}{\\}

\makeatother

\usepackage{listings}
\lstset{frame=single,
basicstyle={\footnotesize\ttfamily},
breaklines=false,
numbers=left,
language=C}
\renewcommand{\lstlistingname}{Listing}

\begin{document}
\title{Debian's Linear Algebra Libraries and Alternatives}
\author{Mo Zhou \\Debian Science Team\\{\tt\small lumin@debian.org}\\License: {\tt CC-BY-SA-4.0}
}
\maketitle
\begin{abstract}
Classical numerical linear algebra libraries, \emph{BLAS} and \emph{LAPACK}
play important roles in the scientific computing field. Varied demand
on these libraries poses non-trivial challenge on system management
and linux distribution development, from which some existing linux
distributions suffer. However, by leveraging dpkg's update-alternatives
mechanism which enables user to switch BLAS and LAPACK libraries smoothly
and painlessly, the problems could be properly and decently addressed.
The mechanism has been validated with several simple experiments,
which illusrate the advancement of the Debian solution. In the end,
the underlying principles are revealed, and some pitfalls are pointed
out.
\end{abstract}
\\ \\
PDF: URL\\
HTML: URL\\
Lyx: URL
\begin{abstract}
\tableofcontents{}
\end{abstract}

\section{Introduction}

{\bf BLAS} (Basic Linear Algebra Subroutines)\cite{blas} and {\bf LAPACK}
(Linear Algebra PACKage)\cite{lapack} are important mathematical
APIs/ABIs/libraries to performance-critical programs that manipulate
dense and contiguous numerical arrays. BLAS provides low-level and
frequently used linear algebra routines for vector and matrix operations;
LAPACK provides higher-level functionality based on complex call graph
over BLAS, hence BLAS is the most performance-critical part. These
libnear algebra libraries are widely used in the scientific computing
and many other areas, for example: Numpy, Scipy, Julia, Octave, R,
Sagemath, Matlab, Mathematica and LINPACK \cite{linpack} \footnote{High performance linpack is used to benchmark and rank supercomputers
for the TOP500 list.}.

In practice, \emph{different} users need \emph{different} BLAS/LAPACK
implementations with \emph{different} configurations (e.g. threading
model or index type variants) under \emph{different} scenarios. Such
extremely diversed and varied BLAS/LAPACK library requirement poses
a non-trivial challenge to linux distribution developers: No single
configuration satisfies all user's demand, e.g. when only the OpenMP
version of OpenBLAS is provided\footnote{This is basically what Archlinux does.},
people who write pthread programs would complain; It leads to confusion
and chaos if many alternative libraries co-exists with different SONAMEs,
e.g. libopenblas, libopenblasp, libopenblaso\footnote{This is actually what Fedora does.};
It is still not the optimal solution to provide the build-time switch,
that would result in a rebuild of the whole reverse dependency tree
when switching \footnote{This is what Gentoo eselect/pkg-config does.};
Forcing distribution packages to use a specific implemtation on a
per-package basis, clearly doesn't satisfy the divsersed user demand
\footnote{This is what BSD/Port does.}.

Taking advantage from the update-alternatives mechanism of dpkg, Debian
is able to address the aforementioned challenge in a decent way. Such
that Debian users could switch the underlying BLAS/LAPACK implementations
without recompiling their applications. At the meantime, many BLAS
implementations with different configurations could co-exist on the
system without confliction. As the outcome, users could easily profile
and benchmark different implementations efficiently, and painlessly
switch the underlying implementation for different programs under
different scenarios.

This article will go through some basics about Linear Algebra libraries,
followed by a brief overview on the current situation of other distributions.
Some simple experiment will be conducted to illustrate the effectiveness
of Debian's solution. At the end of this article, some pitfalls will
be pointed out, and some suggestions will be provided to maintainers
who have packages that depends on BLAS/LAPACK.

Apart from that, If you are not familiar with Linear Algebra, \cite{gs-la}
would be an excellent educational reference. \cite{dsohowto} is an
advanced document on writing good shared libraries.

\section{Background}

In this section, we'll first discuss about some basics of BLAS and
LAPACK. Several non-BLAS/LAPACK linear algebra libraries will be mentioned
next but we'll not dig into them any further. In the end, the situation
of several other linux distributions will be reviewed.

\subsection{Basic Linear Algebra Subroutines (BLAS)}

The standard BLAS reference implementation, netlib BLAS is written
in Fortran. It's API could be grouped into 3 categories: vector-vector,
matrix-vector and matrix-matrix subroutines, which are called level-\{1,2,3\}
subroutines respectively. The Fortran API naming follows a simple
rule: {\tt <type><op>\_}, where ``type'' could be one of ``s''
(single precision float, aka float32), ``d'' (double, float64),
``c'' (single complex) and ``z'' (double complex); ``op'' is
the abbreviated operation name. For example, {\tt sdot\_} refers
to Single precision DOT product (SDOT) between two vectors, which
is a level-1 subroutine; {\tt dgemm\_} refers to Double precision
GEneric Matrix Multiplication (DGEMM), which is a level-3 subroutine.

Fortran's multi-dimentional arrays are stored in column-major\footnote{AFAIK, Fortran and Julia store arrays in column-major order internall.}
order in memory, different from the C language which stores arrays
in row-major order. So it's necessary to mention the BLAS interface
for C language, namely CBLAS, because many modern language that support
Foreign Function Interface (FFI) use the row-major ordering. Similar
to the Fortran BLAS, the naming rule of CBLAS API is {\tt cblas\_<type><op>}
. The C implementation is also provided by netlib BLAS. For more details
such as a complete list of available subroutines please refer to the
documentation of netlib BLAS.

In the fortran code of standard implementation, ``INTEGER'' is used
as the type of multi-dimensional array size and array pointer offsets.
By default the ``INTEGER'' type is ``int32\_t'' on the x86-64
architecture. Accordingly, the CBLAS implementation also use ``int''
as the indexing type. It is obvious that int32\_t is not enough for
extremely large arrays, which doesn't satisfy HPC users' needs. Although
the Fortran code could be compiled to use 64-bit index via the ``-i8''
compiler flag, which turns the default INTEGER type into 64-bit integer,
the C compiler doesn't provide similar magic. Intel-MKL calls the
32-bit index version as LP64 interface (long and pointer are 64bit
wide), and the 64-bit index version as ILP64 interface (int, long
and pointer are 64bit wide). However this term from Fortran may cause
confusion in the C world because LP64/ILP64 could be ambiguous.

\subsubsection{BLAS Implementations}

The optimization of BLAS library involves battling against cache-misses,
leveraging SIMD instructions and parallelization \cite{blas-opt}.
Much endeavor in optimizing BLAS had been made by the industry since
numerical linear algebra is performance critical especially for large
scale problems. An optimized BLAS implementaion is often significantly
faster than the netlib BLAS. To this end, many optimized implementations
exist in the market. Here are some notable implementations:

{\bf Netlib BLAS} is the standard implementation with low-performance
due to the absense of any optimization, written in Fortran77. It provides
the CBLAS interface, which is not a re-implementation in C, but just
a C wrapper over the Fortran routines. This standard implementation
doesn't support multi-threadding, and doesn't provide 64-bit indexing
flavor for its CBLAS part.

{\bf OpenBLAS} is probably the best free-software choice. It is able
to automatically select the optimized code branch according to CPU
capability at the runtime, and usually delivers promising performance.
It provides both BLAS and CBLAS API/ABI. OpenBLAS supports both OpenMP
and pthread multi-threadding, and the user could switch on 64-bit
indexing with ``INTERFACE64=1''.

{\bf BLIS} is a relatively new framework, targeted to advanced users.
BLIS's performance is close to OpenBLAS, but it somtimes perform better
than OpenBLAS or even MKL \cite{blis-perf}. BLIS provides standard
BLAS and CBLAS API and ABI. Note that BLIS runs in single-thread mode
by default \cite{blis-single}. BLIS supports both OpenMP and pthread
multi-threading, and the user could switch on 64-bit indexing.

{\bf Atlas} can be automatically tuned when compiled natively. However
it doesn't provide runtime dispatching according to CPU functionality.
On distributions that care about hardware compatibility it has to
be compiled with the lowest ISA baseline, resulting in extremely poor
performance. (I should ask sylvestre about the past of the package.)

{\bf MKL} is probably the fastest implementation on x86 architecture
featuring many black magics, yet non-free and might be specific to
Intel CPUs. MKL is a huge collection of various math kernel functions,
including but not limited to BLAS and CBLAS. Intel's MKL is more likely
a collection of black magics and here we only talk about the Single
Dynamic Library (SDL), i.e. \verb|libmkl_rt.so|. MKL supports a wide
range of multi-threadding flavors, including Intel/LLVM OpenMP (iomp,
default), GNU OpenMP (gomp), PGI OpenMP (pomp), TBB. 64-bit indexing
is also supported and Intel calls it the ``ilp64'' model. The user
could change MKL's configuration with environment variables. MKL provides
elaborated documentations \cite{mkl}.

{\bf nvBLAS} is a non-free nvidia-GPU-based partial implementation,
i.e. it only implements the level-3 BLAS API. The user have to provide
a configuration file specifying a fallback BLAS library when the target
program is calling level-1 or -2 BLAS subroutines. For large scale
problems, this implementation is usually several magnitude faster
than the CPU-based implementation due to GPU's large scale parallel
architecture. It's another story for tiny scale problems, because
of the latency introduced by hardware interruption (transmitting data
between CPU and GPU).

\subsection{Linear Algebra Package (LAPACK)}

BLAS is frequently called from LAPACK. And there are considerably
complex call graphs for LAPACK subroutines, which implement higher
level functionalities, such as solving systems of linear equations,
linear least square problems, eigenvalue problems, and decomposing
matrices (e.g. singular value decomposition). LAPACK's API naming
schene is similar to BLAS's. For example, \verb|sgesvd_| refers to
Single-precision GEneric Singular Value Decomposition. Please refer
\cite{lapack} or netlib's documentation for complete description.

The standard LAPACK is written in fortran. Due to the same reason
for the existence of CBLAS, there is also a C language counterpart
to LAPACK, named LAPACKE. The LAPACKE API naming rule spiritually
resembles CBLAS'. For example, the \verb|sgesvd_| counterpart is
\verb|LAPACKE_sgesvd|.

\subsubsection{LAPACK Implementations}

Here are some LAPACK implementations:

{\bf Netlib LAPACK} is the standard implementation written in Fortran77.
The C interface LAPACKE is provided by netlib LAPACK. However, different
from CBLAS which is merely a wrapper around fortran subroutines, the
LAPACKE is a native C implementation. The LAPACKE implementation supports
both 32-bit and 64-bit indexing flavors. The Fortran LAPACK could
be compiled to use 64-bit index with the ``-i8'' compiler option.

{\bf OpenBLAS} provides optimized LAPACK and LAPACKE as well. Both
32-bit and 64-bit indexing are supported.

{\bf MKL} the collection of math kernel functions provides both LAPACK
and LAPACKE API/ABI, of course. Intel MKL has strong relationship
with Netlib because Netlib's LAPACKE is written by Intel.

\subsection{Other Linear Algebra Libraries}

The classical BLAS/LAPACK API doesn't meet some developers' or users'
need. Thus, libraries with similar functionality but totally different
APIs were written.

{\bf Eigen}\cite{eigen} is a well-optimized header-only C++ library,
without runtime dispatch according to CPU capability. It's performance
is heavily impacted by the compiler flags, i.e. the ISA baseline.
For example, the same matrix multiplication demo compiled under x86\_64
generic ISA baseline runs many times slower than the one compiled
under skylake (AVX2) ISA baseline. That means for linux distributions
that focus on hardware compatibility must have to compile low-performance
binary executables, and users will silently waste their modern machine's
performance. SIMDebian \cite{simdebian}, a radical Debian partial
fork with radical ISA baseline is proposed to mitigate the performance
waste, because applications can be compiled with e.g. the Skylake
baseline.

{\bf FlexiBLAS} \cite{flexiblas,flexiblas-talk} is a BLAS wrapper
library with runtime exchangable backends, which allows one to exchange
the BLAS implementation at run-time via an environment variable. FlexiBLAS
argues that both the update-alternatives and the LD\_LIBRARY\_PATH
/ LD\_PRELOAD approaches are only applicable for single file implementations,\footnote{This is true. Debian has to deal with this issue by patching upstream
build systems to squash symbols into a single dynamic shared object
file. For example, the Atlas package's BLAS and CBLAS symbols are
squashed in a single libblas.so.3 file, in order to resemble the netlib
libblas.so.3 .}and update-alternatives also requires root permission. This project
also points out deficiencies of static libraries, Gentoo eselect/pkg-config
and BSD ports, in terms of the mess of BLAS/LAPACK linkage.

Discussion about any of them is beyond the scope of this article.

\subsection{Other Linux Distributions}

Packaging for BLAS and LAPACK libraries and their reverse dependencies
have not been gracefully done by other distributions. Or even worse,
the packaging might be a mess.

{\bf Archlinux} chooses to privide only two implementations \cite{arch-openblas-pkgbuild,arch-blas},
the netlib BLAS and OpenBLAS, in order to keep everything simple.
Particularly, the netlib BLAS and OpenBLAS cannot co-exist. The OpenBLAS
package is compiled with OpenMP threading, 32-bit index and without
LAPACK, LAPACKE, or CBLAS. Obviously, developers who write pthread
program or need optimized serial BLAS library will have to compile
BLAS library by themselves (incl. via AUR), and may possibly mess
up the system eventually.

{\bf Fedora} provides many possible configurations to the user, with
different threading model and different indexing types \cite{fedora-openblas-spec}.
For example, all the openblas variants have nearly the same functionality
but they come along with different SONAMEs. The serial variant is
so-named as libopenblas, the openmp variant as libopenblaso, the pthread
variant as libopenblasp, the 64-bit serial variant as libopenblas64,
the 64-bit openmp variant as libopenblaso64, the 64-bit pthread variant
as libopenblasp64. Such design will not only confuse the user, but
also confuse the reverse dependency maintainers.

{\bf Gentoo} provides many implementations to the user. The user
could change the underlying BLAS/LAPACK implementations with ``eselect
switch'' (it switches pkgconfig files). However, there is a significant
defect that every switch would trigger a rebuild of the whole reverse
dependency tree.

{\bf FreeBSD} , although not a Linux distribution, specifies specific
BLAS/LAPACK implementation on a per-package basis, where the package
maintainer decides which exact implementation to use. Apparently this
doesn't satisfy diversed user demands.

{\bf Ubuntu, Mint, Kali, Deepin, ...} belong to the Debian family.
So a theory that holds on Debian could also be applied to them, as
long as there is no significant discrepancy in the version of software
stack. Note, this article is based on Debian Sid (April 2019), which
is basically equivalent to Debian 10 (codenamed ``Buster'').

\section{Debian's Approach}

In this section, the alternatives mechanism will be reviewed first.
In order to make the alternatives system meaningful, different BLAS
and LAPACK implementations must be co-installable, hence the co-installbility
problem is discussed next. Then the way to register different candidates
in the mechanism will be described, followed by a simple usage guide.

\subsection{Dpkg's update-alternatives mechanism}

To be filled up in the future.

\subsection{Co-existence of Various Implementations}

To be filled up in the future.

Including Threading Model \& Index Type Variants

(Serial, OpenMP (GNU, Intel, PGI), Pthread, TBB)

(int32, -i8 (int64))

On Debian, netlib blas produces binary packages including: libblas3,
liblapack3. Package libblas3 ships \$LIBDIR/blas/libblas.so.3 and
provides libblas.so.3 .

Felix Yan wants to know the difference between openblas-cblas and
netlib-cblas, and the difference between openblas-lapack and netlib-lapack.
He also wants to know what would happen when mixing cblas, blas and
lapack from different vendors.

\subsection{Registering Different Candidates}

To be filled up in the future. Please read some debian packaging examples
if you really want to know.

\subsection{Switching Alternatives}

There are two generic ways to switch the underlying libblas.so.3 or
liblapack.so.3 implementation. The first way is via update-alternatives,
which requires root permission. When you have no enough permission,
the second way is to hint ld.so with environment variable LD\_LIBRARY\_PATH
or LD\_PRELOAD.

There are also measures to switch the backend for some specific software.
For example, Numpy users could configure numpy in \verb|~/.numpy-site.cfg|
\cite{numpy-site-cfg}. However, these non-generic tricks are beyond
the scope of this document, hence will not be elaborated.

\subsubsection{Via Update-alternatives}

To be filled up in the future. See \cite{debian-wiki-lalib}

\subsubsection{Via Hinting ld.so}

To be filled up in the future. See \cite{debian-wiki-lalib}

\subsection{Underlying Principles}

To be filled up in the future.

\subsection{Pitfalls \& Failure Cases}
\begin{enumerate}
\item The mixture of OpenMP and pthread (the GIMP trouble?)
\item The mixture of GNU OpenMP (gomp) and Intel OpenMP (iomp) (silent and
incorrect result from MKL)
\item Some packages squashed BLAS and CBLAS ABI into a single shared object.
(recent numpy change)
\item mis-Use 32-bit blas as the 64-bit provider. \cite{archwiki-julia-arpack}
BLAS implementations has to use 32-bit index to resemble the reference
BLAS.
\item Separated BLAS and CBLAS. Numpy bugs: On Debian Atlas provides BLAS
and CBLAS as well, but it delivers BLAS and CBLAS ABIs in separate
shared objects.
\end{enumerate}
When the application calls some non-standard BLAS routines such as
``axpby''. When the application calls openblas\_set\_num\_threads
somehow (julia). BLAS/LAPACK independence is importance if you wish
the mechanism to work properly.

\section{Experiments}

\subsection{Numpy \& BLAS Alternation}

To assess whether the mechanism used by Debian really works, we conduct
a simple experiment with Numpy. I'm not going to exhaust myself, so
extra experiments are postponed.

Test platform: Debian Sid (April 2019) on Thinkpad T470p. Intel I5-7440HQ,
Nvidia 940MX. Alternatives are switched with rover \cite{rover}.
In every experiment where multi-threading could be enabled, I use
4 threads via {\footnotesize{}MKL\_NUM\_THREADS, OPENBLAS\_NUM\_THREADS,
BLIS\_NUM\_THREADS.}{\footnotesize\par}

Numpy is built against BLAS \& LAPACK. \footnote{dpkg -s python3-numpy | rg blas}
We test whether the alternative system is really affecting Numpy's
matrix operation performance by the following simple experiment. The
code will be run against different libblas.so.3 alternatives, and
from the comprarison of elapsed time we could evaluate the validity.

\begin{lstlisting}
#!/usr/bin/python3.7
import time
import numpy as np
TIMES, tm = 10, []
a = np.random.rand(4096, 4096).astype(np.double)
x = np.random.rand(4096, 4096).astype(np.double)
for i in range(TIMES):
    t_start = time.time()
    y = np.matmul(a, x)
    tm.append(time.time() - t_start)
print(f'{np.mean(tm):.3f} pm {np.std(tm):.3f}')
\end{lstlisting}

Results:
\begin{center}
\begin{tabular}{|c|c|}
\hline 
{\small{}Candidate} & {\small{}Elapsed Time (seconds)}\tabularnewline
\hline 
\hline 
{\small{}MKL (iomp)} & {\small{}1.025 pm 0.023}\tabularnewline
\hline 
{\small{}OpenBLAS (pthread)} & {\small{}1.071 pm 0.023}\tabularnewline
\hline 
{\small{}OpenBLAS-Pthread} & {\small{}WIP}\tabularnewline
\hline 
{\small{}OpenBLAS-OpenMP} & {\small{}WIP}\tabularnewline
\hline 
{\small{}OpenBLAS-Serial} & {\small{}WIP}\tabularnewline
\hline 
{\small{}BLIS-OpenMP} & {\small{}1.075 pm 0.037}\tabularnewline
\hline 
{\small{}BLIS-Pthread} & {\small{}1.153 pm 0.086}\tabularnewline
\hline 
{\small{}BLIS-Serial} & {\small{}3.494 pm 0.031}\tabularnewline
\hline 
{\small{}Atlas (serial, generic)} & {\small{}10.410 pm 0.408}\tabularnewline
\hline 
{\small{}Netlib} & {\small{}47.980 pm 1.122}\tabularnewline
\hline 
\end{tabular}
\par\end{center}

Obviously, the standard Netlib implementation is extremely slow and
everyone should avoid actually using it. The experimental results
illustrate that the underlying BLAS backend could be switched smoothly
with the update-alternatives mechanism, and the performance of higher-level
application would be influenced. In this context, no re-complition
is required at all.

\subsection{Messing Up Index Type}

libfoo.c:

{\tiny \begin{verbatim}
#include <stddef.h>
float sasum64(size_t N, const float *X, size_t incX)
{
    float asum = 0.;
    for (size_t i = 0; i < N; i++) {
        asum += (X[i*incX] > 0.) ? X[i*incX] : -X[i*incX];
    }
    return asum;
}
float sasum32(int N, const float *X, int incX)
{
    float asum = 0.;
    for (int i = 0; i < N; i++) {
        asum += (X[i*incX] > 0.) ? X[i*incX] : -X[i*incX];
    }
    return asum;
}
\end{verbatim} }

app.c:

{\tiny \begin{verbatim}
#include <stdio.h>
#include <stddef.h>
float sasum64(int N, const float *X, int incX);
float sasum32(size_t N, const float *X, size_t incX);

int main(void)
{
    float a[] = {1., 2., -3.};
    printf("%f, %f\n", sasum32(3, a, 1), sasum64(3, a, 1));
    return 0;
}
\end{verbatim} }

Objdump libfoo.so:

{\tiny \begin{verbatim}
diff -ru sasum32.asm sasum64.asm --color
--- 32	2019-04-06 05:10:01.640688236 +0000
+++ 64	2019-04-06 05:10:15.645039520 +0000
@@ -1,15 +1,15 @@
 55                   	push   %rbp
 48 89 e5             	mov    %rsp,%rbp
-89 7d ec             	mov    %edi,-0x14(%rbp)
+48 89 7d e8          	mov    %rdi,-0x18(%rbp)
 48 89 75 e0          	mov    %rsi,-0x20(%rbp)
-89 55 e8             	mov    %edx,-0x18(%rbp)
+48 89 55 d8          	mov    %rdx,-0x28(%rbp)
 66 0f ef c0          	pxor   %xmm0,%xmm0
 f3 0f 11 45 fc       	movss  %xmm0,-0x4(%rbp)
-c7 45 f8 00 00 00 00 	movl   $0x0,-0x8(%rbp)
-eb 7c                	jmp    1246 <sasum32+0x9c>
-8b 45 f8             	mov    -0x8(%rbp),%eax
-0f af 45 e8          	imul   -0x18(%rbp),%eax
-48 98                	cltq   
+48 c7 45 f0 00 00 00 	movq   $0x0,-0x10(%rbp)
+00 
+eb 7d                	jmp    1195 <sasum64+0xa0>
+48 8b 45 f0          	mov    -0x10(%rbp),%rax
+48 0f af 45 d8       	imul   -0x28(%rbp),%rax
 48 8d 14 85 00 00 00 	lea    0x0(,%rax,4),%rdx
 00 
 48 8b 45 e0          	mov    -0x20(%rbp),%rax
@@ -17,34 +17,32 @@
 f3 0f 10 00          	movss  (%rax),%xmm0
 66 0f ef c9          	pxor   %xmm1,%xmm1
 0f 2f c1             	comiss %xmm1,%xmm0
-76 1e                	jbe    120d <sasum32+0x63>
-8b 45 f8             	mov    -0x8(%rbp),%eax
-0f af 45 e8          	imul   -0x18(%rbp),%eax
-48 98                	cltq   
+76 1e                	jbe    115b <sasum64+0x66>
+48 8b 45 f0          	mov    -0x10(%rbp),%rax
+48 0f af 45 d8       	imul   -0x28(%rbp),%rax
 48 8d 14 85 00 00 00 	lea    0x0(,%rax,4),%rdx
 00 
 48 8b 45 e0          	mov    -0x20(%rbp),%rax
 48 01 d0             	add    %rdx,%rax
 f3 0f 10 00          	movss  (%rax),%xmm0
-eb 27                	jmp    1234 <sasum32+0x8a>
-8b 45 f8             	mov    -0x8(%rbp),%eax
-0f af 45 e8          	imul   -0x18(%rbp),%eax
-48 98                	cltq   
+eb 27                	jmp    1182 <sasum64+0x8d>
+48 8b 45 f0          	mov    -0x10(%rbp),%rax
+48 0f af 45 d8       	imul   -0x28(%rbp),%rax
 48 8d 14 85 00 00 00 	lea    0x0(,%rax,4),%rdx
 00 
 48 8b 45 e0          	mov    -0x20(%rbp),%rax
 48 01 d0             	add    %rdx,%rax
 f3 0f 10 00          	movss  (%rax),%xmm0
-f3 0f 10 0d cf 0d 00 	movss  0xdcf(%rip),%xmm1        # 2000 <_fini+0xda4>
+f3 0f 10 0d 81 0e 00 	movss  0xe81(%rip),%xmm1        # 2000 <_fini+0xda4>
 00 
 0f 57 c1             	xorps  %xmm1,%xmm0
 f3 0f 10 4d fc       	movss  -0x4(%rbp),%xmm1
 f3 0f 58 c1          	addss  %xmm1,%xmm0
 f3 0f 11 45 fc       	movss  %xmm0,-0x4(%rbp)
-83 45 f8 01          	addl   $0x1,-0x8(%rbp)
-8b 45 f8             	mov    -0x8(%rbp),%eax
-3b 45 ec             	cmp    -0x14(%rbp),%eax
-0f 8c 78 ff ff ff    	jl     11ca <sasum32+0x20>
+48 83 45 f0 01       	addq   $0x1,-0x10(%rbp)
+48 8b 45 f0          	mov    -0x10(%rbp),%rax
+48 3b 45 e8          	cmp    -0x18(%rbp),%rax
+0f 82 75 ff ff ff    	jb     1118 <sasum64+0x23>
 f3 0f 10 45 fc       	movss  -0x4(%rbp),%xmm0
 5d                   	pop    %rbp
 c3                   	retq   
\end{verbatim} }

Objdump app:

{\tiny \begin{verbatim}
    1184:	48 8d 45 f4          	lea    -0xc(%rbp),%rax
    1188:	ba 01 00 00 00       	mov    $0x1,%edx
    118d:	48 89 c6             	mov    %rax,%rsi
    1190:	bf 03 00 00 00       	mov    $0x3,%edi
    1195:	e8 a6 fe ff ff       	callq  1040 <sasum64@plt>
    119a:	f3 0f 5a d0          	cvtss2sd %xmm0,%xmm2
    119e:	f2 0f 11 55 e8       	movsd  %xmm2,-0x18(%rbp)
    11a3:	48 8d 45 f4          	lea    -0xc(%rbp),%rax
    11a7:	ba 01 00 00 00       	mov    $0x1,%edx
    11ac:	48 89 c6             	mov    %rax,%rsi
    11af:	bf 03 00 00 00       	mov    $0x3,%edi
    11b4:	e8 97 fe ff ff       	callq  1050 <sasum32@plt>
\end{verbatim} }

Or maybe better analyze this with gcc -S?

\section{Conclusion}

Formal conclusion has been omitted. If I must draw some conclusions
here, I would say: \emph{Debian is probably the best GNU/Linux distribution}.

\appendix

\section{To Debian Maintainers}

\section{Porting the Mechanism to Other Distros}

Archlinux: Nobody working on it. Fedora: Nobody.

\subsection{Porting to Gentoo}

Planned as a GSoC 2019 Project \cite{gentoo-gsoc}, initiated by Gentoo.
``BLAS and LAPACK runtime switching''

\subsubsection{Abstract}

Classical numerical linear algebra libraries, \emph{BLAS} and \emph{LAPACK}
play important roles in the scientific computing field. Varied demand
on these libraries poses non-trivial challenge on system management
and linux distribution development, from which some existing linux
distributions suffer. However, by leveraging dpkg's update-alternatives
mechanism which enables user to switch BLAS and LAPACK libraries smoothly
and painlessly, the problems could be properly and decently addressed.
This project aims at introducing the mechanism into Gentoo's BLAS
and LAPACK package dependency tree, addressing multiple existing problems.

\subsubsection{Rationale}

{\bf BLAS} (Basic Linear Algebra Subroutines)\cite{blas} and {\bf LAPACK}
(Linear Algebra PACKage)\cite{lapack} are important mathematical
APIs/ABIs/libraries to performance-critical programs that manipulate
dense and contiguous numerical arrays. BLAS provides low-level and
frequently used linear algebra routines for vector and matrix operations;
LAPACK provides higher-level functionality based on complex call graph
over BLAS, hence BLAS is the most performance-critical part. These
libnear algebra libraries are widely used in the scientific computing
and many other areas, for example: Numpy, Scipy, Julia, Octave, R,
Sagemath, Matlab, Mathematica and LINPACK \cite{linpack} \footnote{High performance linpack is used to benchmark and rank supercomputers
for the TOP500 list.}.

In practice, \emph{different} users need \emph{different} BLAS/LAPACK
implementations with \emph{different} configurations (e.g. threading
model or index type variants) under \emph{different} scenarios. Such
extremely diversed and varied BLAS/LAPACK library requirement poses
a non-trivial challenge to linux distribution developers: No single
configuration satisfies all user's demand, e.g. when only the OpenMP
version of OpenBLAS is provided\footnote{This is basically what Archlinux does.},
people who write pthread programs would complain; It leads to confusion
and chaos if many alternative libraries co-exists with different SONAMEs,
e.g. libopenblas, libopenblasp, libopenblaso\footnote{This is actually what Fedora does.};
It is still not the optimal solution to provide the build-time switch,
that would result in a rebuild of the whole reverse dependency tree
when switching \footnote{This is what Gentoo eselect/pkg-config does.};
Forcing distribution packages to use a specific implemtation on a
per-package basis, clearly doesn't satisfy the divsersed user demand
\footnote{This is what BSD/Port does.}.

Taking advantage from the update-alternatives mechanism of dpkg, Debian
is able to address the aforementioned challenge in a decent way. Such
that Debian users could switch the underlying BLAS/LAPACK implementations
without recompiling their applications. At the meantime, many BLAS
implementations with different configurations could co-exist on the
system without confliction. As the outcome, users could easily profile
and benchmark different implementations efficiently, and painlessly
switch the underlying implementation for different programs under
different scenarios.

The solution of ``BLAS and LAPACK runtime switching'' is to mimic
Debian. The Netlib BLAS and LAPACK packages are mandatory when one
needs to build any application on top of BLAS/LAPACK. The netlib implementation
functions as the last fallback implementation, a linkder stub, the
standard header provider, and most importantly the fallback alternative
for libblas.so.3 and liblapack.so.3. All packages depend on BLAS and
LAPACK should link against libblas.so.3 or liblapack.so.3, which could
be provided by various BLAS and LAPACK implementations. The recommended
header used to build reverse dependencies is the one from netlib.
All other BLAS and LAPACK implementations should register themselves
as alternatives to the netlib implementation, where each implementation
provides two groups of candidates: development files, and shared objects.

\subsubsection{Objective}
\begin{enumerate}
\item Integrate update-alternatives mechanism into Gentoo's reference blas,
i.e. netlib blas' packaging. Or update the eslect-blas integration.
\item Package BLIS for Gentoo, and register it as another netlib blas alternative.
\item Update the packaging of a BLAS/LAPACK reverse dependency to validate
the efficacy of the mechanism on Gentoo. Specifically, the package
will be compiled against BLIS's libblas.so.3 and netlib's libblas.so.3
respectively, and always use the netlib headers.
\item Modify the packaging for OpenBLAS to register libopenblas.so as an
alternative to netlib blas. Header are not exposed in public include
directory.
\item Update packaging for BLAS and LAPACK reverse dependencies, and enforce
linkage against libblas.so.3 (any BLAS implementation could provide
this).
\item Update gentoo wiki or documentations, describing related changes.
\item Optionally, write an eselect-to-update-alternatives bridge, if the
implementation used u-a in the end.
\end{enumerate}

\subsubsection{Deliverables}
\begin{enumerate}
\item Updated/NEW gentoo packaging for at least Netlib, OpenBLAS and BLIS.
\item Updated BLAS/LAPACK reverse dependency packaging for at least the
most important ones.
\item Documentation/wiki updates regarding the updates.
\item Optionally an eselect-to-update-alternatives bridge.
\end{enumerate}

\subsubsection{Timeline}

The official GSoC time frame is May 27, 2019 - August 19, 2019. The
following timeline stick to the official time frame. However, in order
to avoid confliction with my other activities, I may start the project
early.

May 27 - June 10 (2 weeks): Get familiar with Gentoo development and
update packaging for blas-reference, cblas-reference.

June 10 - June 17 (1 week): Package BLIS for Gentoo, and register
it as netlib alternative.

June 17 - June 24 (1 week): Update the packaging for Numpy and Octave,
and conduct sanity tests regarding the mechanism.

June 24 - July 1 (1 week): Modify the OpenBLAS packaging and register
it in the mechanism. Of course, sanity tests should be conducted.

July 1 - July 15 (2 weeks): Update packaging for several other important
BLAS and LAPACK reverse dependencies.

July 15 - July 22 (1 week): Write documentation regarding to the changes,
and suggestion to BLAS/LAPACK reverse dependency package maintainers.

July 22 - Aug 5 (2 weeks): Try to implement an eselect-to-update-alternatives
bridge.

Aug 5 - Aug 19 (2 weeks): Update as many BLAS / LAPACK reverse dependency
packages as possible.

\section*{Acknowledgement}

The development is funded by myself, and I'm writting this article
as an excercise.

Coping \& Pasting software from the internet to the archive, such
repetitive pattern will eventually burn out one's interest in Debian
development. Only by doing works that make difference can I feel that
I'm really ``developing'' Debian, despite the increased difficulty.

This work could not have been done without my fellow developers' historical
works: S�bastien Villemot, Sylvestre Ledru, and the DPKG developers.

Thanks to Felix Yan for everything related to Archlinux; Benda Xu
for everything related to Gentoo.
\begin{thebibliography}{10}
\bibitem{gs-la} Gilbert Strang, Introduction to Linear Algebra, \url{http://math.mit.edu/~gs/linearalgebra/}

\bibitem{eigen} Eigen3, http://eigen.tuxfamily.org

\bibitem{simdebian} SIMDebian, https://github.com/SIMDebian/SIMDebian

\bibitem{fedora-openblas-spec} https://src.fedoraproject.org/cgit/rpms/openblas.git/
tree/openblas.spec

\bibitem{arch-openblas-pkgbuild}https://git.archlinux.org/svntogit/community.git/
tree/trunk/PKGBUILD?h=packages/openblas

\bibitem{blis-single} https://github.com/flame/blis/issues/292

\bibitem{rover} Rover: https://salsa.debian.org/debian/rover

\bibitem{numpy-site-cfg} https://github.com/numpy/numpy/blob/master/site.cfg.example

\bibitem{blas-opt} https://github.com/flame/how-to-optimize-gemm

\bibitem{gentoo-gsoc} https://wiki.gentoo.org/wiki/Google\_Summer\_of\_Code/2019/
Ideas\#BLAS\_and\_LAPACK\_runtime\_switching

\bibitem{archwiki-julia-arpack} https://wiki.archlinux.org/index.php/Julia\#Arpack

\bibitem{openblas-64bit-default} https://github.com/xianyi/OpenBLAS/issues/1763

\bibitem{dsohowto} Ulrich Drepper, How To Write Shared Libraries

\bibitem{mkl} https://software.intel.com/en-us/mkl

\bibitem{blas} https://en.wikipedia.org/wiki/Basic\_Linear\_Algebra\_Subprograms

\bibitem{lapack} https://en.wikipedia.org/wiki/LAPACK

\bibitem{linpack} https://en.wikipedia.org/wiki/LINPACK

\bibitem{debian-wiki-lalib} https://wiki.debian.org/DebianScience/LinearAlgebraLibraries

\bibitem{blis-perf} https://github.com/flame/blis/blob/master/docs/Performance.md

\bibitem{flexiblas} https://www.mpi-magdeburg.mpg.de/projects/flexiblas

\bibitem{flexiblas-talk} https://www2.mpi-magdeburg.mpg.de/mpcsc/software/flexiblas/2018\_talk\_fin\_flexiblas.pdf

\bibitem{arch-blas} https://www.archlinux.org/packages/extra/x86\_64/blas/
\end{thebibliography}

\end{document}
