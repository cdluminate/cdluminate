%% LyX 2.3.3 created this file.  For more info, see http://www.lyx.org/.
%% Do not edit unless you really know what you are doing.
\documentclass[twocolumn,english]{article}
\usepackage{mathptmx}
\renewcommand{\ttdefault}{cmtt}
\usepackage[T1]{fontenc}
\usepackage[latin9]{inputenc}
\usepackage{geometry}
\geometry{verbose,tmargin=2.5cm,bmargin=2.5cm,lmargin=2cm,rmargin=2cm}
\setcounter{tocdepth}{2}
\usepackage{babel}
\usepackage{listings}
\lstset{frame=single,
basicstyle={\footnotesize\ttfamily},
breaklines=false,
numbers=left,
language=C}
\begin{document}
\title{BLAS and LAPACK runtime switching}
\author{Mo Zhou\\lumin@debian.org}
\maketitle
\begin{abstract}
Classical numerical linear algebra libraries, BLAS and LAPACK play
important roles in the scientific computing field. Various demands
on these libraries pose non-trivial challenge on system management
and linux distribution development. By leveraging Debian's update-alternatives
mechanism which enables user to switch BLAS and LAPACK libraries smoothly
and painlessly, the problems could be properly and decently addressed.
This project aims at introducing the mechanism into Gentoo\textquoteright s
eselect framework to manage BLAS and LAPACK, providing equivalent
or better functionality of Debian's update-alternatives.
\end{abstract}

\section{Rationale}

BLAS (Basic Linear Algebra Subroutines)\cite{blas} and LAPACK (Linear
Algebra PACKage)\cite{lapack} are important mathematical APIs/ABIs/libraries
to performance-critical programs that manipulate dense and contiguous
numerical arrays. BLAS provides low-level and frequently used linear
algebra routines for vector and matrix operations; LAPACK provides
higher-level functionality based on complex call graph over BLAS,
hence BLAS is the most performance-critical part. These libnear algebra
libraries are widely used in the scientific computing and many other
areas, for example: Numpy, Scipy, Julia, Octave, R, Sagemath, Matlab,
Mathematica and LINPACK \cite{linpack} (High performance linpack
is used to benchmark and rank supercomputers for the TOP500 list).

In practice, different users need different BLAS/LAPACK implementations
with different configurations (e.g. threading model or index type
variants) under different scenarios. Such extremely diversed and varied
BLAS/LAPACK library requirement poses a non-trivial challenge to linux
distribution developers.

Taking advantage from the update-alternatives mechanism of dpkg, Debian
is able to address the challenge in a decent way. Debian users could
switch the underlying BLAS/LAPACK implementations without recompiling
their applications. At the same time, many BLAS implementations with
different configurations could co-exist on the system without confliction.
As a result, users could easily profile and benchmark different implementations
efficiently, and painlessly switch the underlying implementation for
different programs under different scenarios.

\subsection{Solutions of Other Distros}

Archlinux only provides the Netlib and OpenBLAS, where they conflict
to each other. This won't satisfy all user's demand, because the OpenMP
version of OpenBLAS is not a good choice for pthread-based programs.

Fedora's solution is to provide every possible version, and make them
co-installable by assigning different SONAMEs. It leads to confusion
and chaos if many alternative libraries co-exists on the system, e.g.
libopenblas, libopenblasp, libopenblaso.

The BSD (Port) Family forces packages to use a specific implemtation
on a per-package basis, which clearly doesn't satisfy the divsersed
user demand.

\subsection{Gentoo's Solution}

Currently Gentoo's solution is based on eselect/pkg-config.
\begin{enumerate}
\item Gentoo's historical ``alternatives.eclass'' and the present ``alternatives-2.eclass''
are designed to mimic Debian's update-alternatives mechanism.
\item ``alternatives-2.eclass'' is still being tested and experimented
in the science-overlay. Is still cannot be merged into Gentoo main
repo because it doesn't fully comply with the requirements. Specifically,
the eselect mechanism should not infect the build-time behaviour.
This project will possibly reuse the code of ``alternatives.eclass''
and ``alternatives-2.eclass''.
\item The Gentoo community is still debating\cite{key-1} on how to select
BLAS and LAPACK. Some people think the update-alternatives mechanism
is more flexible than the eselect mechanism, and hence Gentoo should
adopt Debian's runtime-switch mechanism. Others argue that recompiling
is inevitable after switch.
\end{enumerate}

\subsection{Proposed Solution}

The solution of ``BLAS and LAPACK runtime switching'' is to mimic
Debian, based on the eselect mechanism. The Netlib BLAS and LAPACK
packages are mandatory when one needs to build any application on
top of BLAS/LAPACK. The netlib implementation functions as the fallback
implementation, a linkder stub, the standard header provider, and
most importantly the fallback alternative for libblas.so.3 and liblapack.so.3.
All packages depending on BLAS and LAPACK should link against libblas.so.3
and liblapack.so.3 respectively. They in turn could be provided by
various 3rd party BLAS and LAPACK implementations. The recommended
header used to build reverse dependencies is the one from netlib.
All other BLAS and LAPACK implementations should register themselves
as alternatives to the netlib implementation. Each implementation
provides candidates for the shared objects, and their development
files are stored in their own subdirectory.

\section{Objective}

Enable the users to switch the BLAS and LAPACK backend without re-compiling
the reverse dependendies.

\section{Deliverables}
\begin{enumerate}
\item Develop eslect-blas and eselect-lapack integrations, with reference
to Debian's update-alternatives, and the previous alternatives\{,-2\}.eclass.
\item Package BLIS for Gentoo, and register it as another netlib blas alternative.
\item Update the packaging of a BLAS/LAPACK reverse dependency to validate
the efficacy of the mechanism on Gentoo. Specifically, the package
will be compiled against BLIS's libblas.so.3 and netlib's libblas.so.3
respectively, and always use the netlib headers.
\item Modify the packaging for OpenBLAS to register libopenblas.so as an
alternative to netlib blas. Header are not exposed in public include
directory.
\item Update packaging for BLAS and LAPACK reverse dependencies, including
Numpy, Scipy and R and enforce linkage against libblas.so.3 (any BLAS
implementation could provide this). Note, Julia must be compiled against
OpenBLAS. So Julia will also be tested to see whether the mechanism
will have broken special packages such as Julia.
\item Upload (NEW) gentoo packaging for at least Netlib, OpenBLAS and BLIS.
\item Officialize BLAS/LAPACK reverse dependency packages for at least the
most important ones, e.g. Numpy, Scipy, R and Julia.
\item Update gentoo wiki or documentations, describing related changes.
\end{enumerate}

\section{Timeline}

I will be able to work full time during the ofcial GSoC time frame
May 27, 2019 -August 19. 2019.
\begin{enumerate}
\item May 6 - May 27 (3 weeks): Get familiar with Gentoo development. Get
to know the Gentoo science team.
\item May 27 - June 10 (2 weeks): Update packaging for blas-reference, cblas-reference.
\item June 10 - June 17 (1 week): Package BLIS for Gentoo, and register
it as netlib alternative.
\item June 17 - June 24 (1 week): Update the packaging for several important
packages such as Numpy, R and Julia, and conduct sanity tests regarding
the mechanism.
\item June 24 - July 1 (1 week): Modify the OpenBLAS packaging and register
it in the mechanism. Sanity tests will be carried out as well.
\item July 1 - July 22 (3 weeks): Update packaging for other important BLAS
and LAPACK reverse dependencies, e.g. Octave and Scipy, and conduct
sanity test.
\item July 22 - July 29 (1 week): Write documentation regarding to the changes.
\item Aug 5 - Aug 19 (2 weeks): Work with the Gentoo science team to stardardize
the new mechanism and merge selected packages of the science overlay
into Gentoo main repository. Fix all the rest BLAS/LAPACK reverse
dependencies.
\end{enumerate}
\begin{thebibliography}{1}
\bibitem{blas} https://en.wikipedia.org/wiki/Basic\_Linear\_Algebra\_Subprograms

\bibitem{lapack} https://en.wikipedia.org/wiki/LAPACK

\bibitem{linpack} https://en.wikipedia.org/wiki/LINPACK

\bibitem{key-1} https://github.com/gentoo/sci/issues/805

\bibitem{key-2} https://nm.debian.org/person/lumin

\bibitem{key-3} https://qa.debian.org/developer.php?login=lumin
\end{thebibliography}

\end{document}
