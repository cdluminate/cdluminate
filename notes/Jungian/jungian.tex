\documentclass[UTF8,letter,10pt]{ctexart}
\usepackage{geometry}
\geometry{left=2.5cm, right=2.5cm, top=2.5cm, bottom=2.5cm}
\setmainfont{Times New Roman}
\usepackage{booktabs}
\usepackage{colortbl}
\usepackage{xcolor}
\usepackage{microtype}
\usepackage{multirow}
\usepackage{makecell}
\usepackage{graphicx}
\usepackage{float}
\usepackage{hyperref}

\definecolor{intjpurple}{HTML}{88619A}
%\definecolor{intjpurple}{HTML}{4A148C}


\begin{document}

\begin{table}[h!]
    \caption{Cheatsheet索引表}
    \centering
    \begin{tabular}{cl}
        \toprule
        链接 & 表格名称 \\
        \midrule
        \ref{tab:mbti-jung} & MBTI - 荣格八维映射表 (Function Stacks)  \\
        \ref{tab:cog-function-roles} & 认知功能的角色表(含标准定义符合度) \\
        \ref{tab:cog-function-normal} & 认知功能的正常表现 \\
        \ref{tab:cog-function-child} & 认知功能的3*永恒之子(Child)表现 \\
        \ref{tab:cog-function-inferior} & 认知功能的4*阿尼玛(Inferior)表现 \\
        \ref{tab:cog-function-inferior-remedy} & 极端压力/抓取状态(Grip)的解药 \\
        \ref{tab:cog-function-critical} & 认知功能的6*批判家长(Critical Parents)表现 \\
        \ref{tab:cog-function-trickster} & 认知功能的7*盲点(Trickster)表现 \\
        \ref{tab:cog-function-demon} & 认知功能的8*恶魔(Demon)表现 \\
        \ref{tab:cog-function-demon-redemption} & 认知功能的8*恶魔(Demon)整合 \\
        \ref{tab:cog-function-loop} & 认知功能的防御机制/长期内耗循环(Loop) \\
        \ref{tab:cog-function-loop-remedy} & 长期内耗循环(Loop)的解药 \\
        \bottomrule
    \end{tabular}
\end{table}


\begin{table}[h!]
    \caption{MBTI - 荣格八维映射表 (Function Stacks)}
    \label{tab:mbti-jung}
    \centering
    \begin{tabular}{cccc}
        \toprule
        \multicolumn{4}{c}{\textbf{NT -- 分析师 Analysts}} \\
        \cellcolor{intjpurple}\color{white} INTJ (Ni-Te-Fi-Se) & INTP (Ti-Ne-Si-Fe) & ENTJ (Te-Ni-Se-Fi) & ENTP (Ne-Ti-Fe-Si) \\
        \midrule
        \multicolumn{4}{c}{\textbf{NF -- 外交官 Diplomats}} \\
        INFJ (Ni-Fe-Ti-Se) & INFP (Fi-Ne-Si-Te) & ENFJ (Fe-Ni-Se-Ti) & ENFP (Ne-Fi-Te-Si) \\
        \midrule
        \multicolumn{4}{c}{\textbf{SJ -- 守护者 Sentinels}} \\
        ISTJ (Si-Te-Fi-Ne) & ISFJ (Si-Fe-Ti-Ne) & ESTJ (Te-Si-Ne-Fi) & ESFJ (Fe-Si-Ne-Ti) \\
        \midrule
        \multicolumn{4}{c}{\textbf{SP -- 探索者 Explorers}} \\
        ISTP (Ti-Se-Ni-Fe) & ISFP (Fi-Se-Ni-Te) & ESTP (Se-Ti-Fe-Ni) & ESFP (Se-Fi-Te-Ni) \\
        \bottomrule
    \end{tabular}
\end{table}


\begin{table}[h!]
    \caption{认知功能的角色表(含标准定义符合度)}
    \label{tab:cog-function-roles}
    \centering
    \resizebox{\textwidth}{!}{%
    \begin{tabular}{ccccc}
    \toprule
    序号 & 位置 & 角色名称 & 典型表现 & 对标“正常表现”程度\tabularnewline
    \midrule
        1 & 主导 (Dominating) & 英雄(Hero) & 核心力量 & 120\% (本能超载:如呼吸般自然)\tabularnewline
        2 & 辅助 (Auxiliary) & 家长(Parent) & 辅助工具 & 100\% (完美对标:最成熟可控)\tabularnewline
        3 & 第三 (Tertiary) & 孩子(Child) & 舒适区 & 50\% $\sim$ 70\% (任性/游戏:表现不稳定)\tabularnewline
        4 & 劣势 (Inferior) & 阿尼玛(Anima) & 压力源泉/理想 & 10\% (原始) $\rightarrow$ 70\% (整合后工具化)\tabularnewline
    \midrule
        5 & 影子 & 反向作用(Opposing) & 质疑者 & -20\% (抗拒/相反:态度厌恶)\tabularnewline
        6 & 影子 & 批判家长(Critical) & 挑刺者 & -50\% (双标/虚伪:严于律人)\tabularnewline
    \rowcolor{gray!17}7 & 影子 & 盲点(Trickster) & 真正盲区 & 0\% (黑洞/混乱:无法理解)\tabularnewline
        8 & 影子 & 恶魔(Demon) & 毁灭者 & -100\% (彻底颠覆:旨在摧毁)\tabularnewline
    \bottomrule
    \end{tabular}}
\end{table}


\begin{table}[h!]
    \caption{认知功能的标准健康表现(The Ideal State)。
这通常是1*主导(Hero)与2*辅助(Parent)功能的天然状态。
在个体化(Individuation)的高阶整合中,即便是劣势功能,通过刻意练习也能呈现出此表的特质。
\textbf{核心区别}:原生功能是自动化的“充电器”(Charging),而通过后天训练整合的功能——尽管外部表现完美——本质上仍是需要消耗专注力的“耗电器”(Draining)。}
    \label{tab:cog-function-normal}
    \centering
\begin{tabular}{cccc}
\toprule
简称 & 全称 & 核心词 & 正常表现\tabularnewline
\midrule
Se & Extraverted Sensing & 体验,当下,刺激 & 活在当下,直接行动,现实主义\tabularnewline
Si & Introverted Sensing & 经验,细节,稳健 & 参照经验,细节敏感,追求守序\tabularnewline
Ne & Extraverted Intuition & 发散,脑洞,可能 & 跳跃思维,探索新奇,抽象关联\tabularnewline
Ni & Introverted Intuition & 收敛,洞察,本质 & 预测未来,现象本质,单一愿景\tabularnewline
Te & Extraverted Thinking & 效率,执行,调配 & 结果导向,逻辑外化,强势直接\tabularnewline
Ti & Introverted Thinking & 逻辑,解构,精准 & 追求真理,刨根问底,无视权威\tabularnewline
Fe & Extraverted Feeling & 和谐,人际,共情 & 情绪传染,维护关系,社交礼仪\tabularnewline
Fi & Introverted Feeling & 本心,真实,价值 & 忠于自我,情感内化,排斥虚伪\tabularnewline
\bottomrule
\end{tabular}

\end{table}


\begin{table}[h!]
\caption{认知功能的3*永恒之子(Child)表现: 治愈、游戏与任性。这是人格的“游乐场”与退行区。表现形式:天真单纯的创造力(Playfulness),或是压力下像被宠坏的孩子般无理取闹(Immaturity)。}
    \label{tab:cog-function-child}
    \centering
    \resizebox{\textwidth}{!}{
\begin{tabular}{ccccc}
\toprule
孩童功能 & 涉及类型 & 核心台词 & 游戏表现 (Playfulness) & 幼稚表现 (Immaturity)\tabularnewline
\midrule
\multicolumn{5}{c}{理性游戏组:Judgement}\tabularnewline
\midrule
Ti  & INFJ, ISFJ & 居然是这样! & 纯粹好奇,钻研冷门,解谜乐趣 & 逻辑死板,钻牛角尖,固执己见\tabularnewline
Fi  & INTJ, ISTJ & 我偏要喜欢 & 傲娇(Tsundere),深情忠诚,情感纯粹 & 情感任性,无视大局,敏感记仇\tabularnewline
Te  & ENFP, ESFP & 我能搞定! & 虚张声势,突发指挥,热衷整理 & 盲目自信,拒绝援助,三分钟热度\tabularnewline
Fe  & ENTP, ESTP & 大家都爱我 & 迷人捣蛋,幽默风趣,活跃气氛 & 渴望关注,哗众取宠,惧怕被弃\tabularnewline
\midrule
\multicolumn{5}{c}{感知游戏组:Perception}\tabularnewline
\midrule
Ni  & ISTP, ISFP & 我有预感... & 迷信直觉,神秘主义,宿命感 & 盲信猜测,无端幻想,消极宿命\tabularnewline
Si  & INTP, INFP & 还是旧的好 & 恋旧收藏,沉溺舒适,重复安稳 & 抗拒改变,退行童年,逃避现实\tabularnewline
Ne  & ESTJ, ESFJ & 如果这样呢? & 突发脑洞,冷笑话,无厘头创意 & 容易分心,盲目乐观,无法落地\tabularnewline
Se  & ENTJ, ENFJ & 我全都要! & 报复享乐,追求奢华,感官盛宴 & 挥霍无度,暴饮暴食,缺乏节制\tabularnewline
\bottomrule
\end{tabular}}
\end{table}


\begin{table}[h!]
\caption{认知功能的4*阿尼玛(Inferior)表现: 渴望与劫持(The Grip)。这是人格的“阿喀琉斯之踵”,也是通往潜意识的桥梁。表现形式:平时极度崇拜拥有此功能的人(Aspiration),但在极度压力或疲惫下,则会变成拙劣的模仿者或暴君,导致毁灭性的抓取状态。}
    \label{tab:cog-function-inferior}
\centering
\resizebox{\textwidth}{!}{
\begin{tabular}{ccccc}
\toprule
劣势功能 & 涉及类型 & 核心渴望 (Aspiration) & 压力触发 (Trigger) & 抓取表现 (The Grip)\tabularnewline
\midrule
\multicolumn{5}{c}{感知型阿尼玛:Perception}\tabularnewline
\midrule
Se & INTJ, INFJ & 自由与现实掌控 & 脑力透支,现实失控 & 感官放纵,强迫行为,暴饮暴食\tabularnewline
Si & ENTP, ENFP & 安稳与归属感 & 细节淹没,身体不适 & 疑病退缩,钻牛角尖,强迫守旧\tabularnewline
Ne & ISTJ, ISFJ & 智慧与变通性 & 计划打乱,未知变故 & 灾难思维,阴谋论,过度联想\tabularnewline
Ni & ESTP, ESFP & 深刻与人生意义 & 未来无望,理论困扰 & 宿命论,被害妄想,黑暗解读\tabularnewline
\midrule
\multicolumn{5}{c}{判断型阿尼玛:Judgement}\tabularnewline
\midrule
Te & INFP, ISFP & 高效成就与秩序 & 价值受挫,自觉无能 & 暴君模式,急功近利,归咎他人\tabularnewline
Ti & ENFJ, ESFJ & 逻辑真理与独立 & 人际破裂,逻辑无效 & 冷酷逻辑,过度分析,自我贬低\tabularnewline
Fe & INTP, ISTP & 情感接纳与和谐 & 遭忽视,情感冲突 & 情绪失控,受害心态,过度敏感\tabularnewline
Fi & ENTJ, ESTJ & 内心平和与真实 & 权威挑战,计划失败 & 情感泛滥,自怜抑郁,被动攻击\tabularnewline
\bottomrule
\end{tabular}}
\end{table}


\begin{table}[h!]
\caption{极端压力/抓取状态(Grip)的解药。当主导功能枯竭时,切忌(用主导功能)硬抗,此时唯一的解药是安抚劣势功能,顺应其最原始的需求。}
    \label{tab:cog-function-inferior-remedy}
    \centering
    \resizebox{\textwidth}{!}{
\begin{tabular}{ccccc}
\toprule
劣势功能 & 涉及类型 & 核心策略 (Strategy) & 停止行为 (Stop) & 执行清单 (Action)\tabularnewline
\midrule
\multicolumn{5}{c}{感知解药:回归生理与当下 (Perception)}\tabularnewline
\midrule
Se & INTJ, INFJ & \textbf{低压感官疗法} & 停止挖掘意义,停止死磕理论 & 高质量睡眠,接触大自然,摄影,无目的散步\tabularnewline
Si & ENTP, ENFP & \textbf{熟悉感与滋养} & 停止探索新世界,停止焦虑未来 & 吃熟悉的舒适食物,温水澡,简化待办,回安全屋\tabularnewline
Ne & ISTJ, ISFJ & \textbf{现实检验锚点} & 停止灾难化想象,停止过度推演 & 罗列已知事实,找人分担,做微小但确定的改变\tabularnewline
Ni & ESTP, ESFP & \textbf{放过明天} & 停止当下反应,停止试图解决未来 & 推迟所有决策,借用他人愿景,纯粹的娱乐转移\tabularnewline
\midrule
\multicolumn{5}{c}{判断解药:回归秩序与人性 (Judgement)}\tabularnewline
\midrule
Te & INFP, ISFP & \textbf{小赢策略} & 停止宏大目标,停止自我攻击 & 整理一个抽屉,划掉简单任务,整理房间,承认无力\tabularnewline
Ti & ENFJ, ESFJ & \textbf{逻辑隔离} & 停止情绪读心,停止人际纠结 & 独处充电,玩解谜/数字游戏,做不涉及人的智力活动\tabularnewline
Fe & INTP, ISTP & \textbf{无压陪伴} & 停止彻底隔离,停止逻辑辩论 & 并在但各玩各的(Parallel Play),书面沟通,身体接触\tabularnewline
Fi & ENTJ, ESTJ & \textbf{允许脆弱} & 停止压抑分析,停止强撑坚强 & 私密环境哭泣宣泄,暂停工作,寻找安全盟友倾诉\tabularnewline
\bottomrule
\end{tabular}}
\end{table}


\begin{table}
\caption{认知功能的6*批判家长(Critical Parents)表现: 如何打压别人和自己}
    \label{tab:cog-function-critical}
    \resizebox{\textwidth}{!}{
\begin{tabular}{ccccc}
\toprule
批判功能 & 涉及类型 & 核心台词 & 对外打压 & 对内自毁\tabularnewline
\midrule
\multicolumn{5}{c}{理性审判组:Judgement}\tabularnewline
\midrule
Ti Critical 批判逻辑 & INTJ, ISTJ & 你真蠢 & 智力羞辱,冷漠指出他人逻辑漏洞 & 自我怀疑,质疑自己不够聪明\tabularnewline
Fi Critical 批判道德 & INFJ, ISFJ & 你真坏 & 道德审判,指责自私虚伪冷血人品 & 自我羞耻,觉得自己是伪君子\tabularnewline
Te Critical 批判能力 & ENTP, ESTP & 你没用 & 能力否定,嘲笑笨手笨脚一事无成 & 成就焦虑,成功仍觉一无是处\tabularnewline
Fe Critical 批判社交 & ENFP, ESFP & 没人理你 & 社交孤立,攻击对方情商或者礼仪 & 归属崩塌,总觉自己格格不入\tabularnewline
\midrule
\multicolumn{5}{c}{感知审判组:Perception}\tabularnewline
\midrule
Ni Critical 批判远见 & INTP, INFP & 没希望 & 未来诅咒,否定愿景断言注定失败 & 虚无主义,觉得做什么都没意义\tabularnewline
Si Critical 批判经验 & ISTP, ISFP & 你真弱 & 翻旧帐,指责不可靠不长记性/否定毅力 & 创伤反刍,不断回忆过去的痛苦\tabularnewline
Ne Critical 批判意图 & ENTJ, ENFJ & 你想害我 & 动机阴谋论,怀疑忠诚指责对方鲁莽 & 选择障碍,看到无数会感到受困\tabularnewline
Se Critical 批判现实 & ESTJ, ESFJ & 真难看 & 外貌/品味攻击,苛刻挑剔对方衣着审美 & 容貌/现实焦虑,觉得自己丑陋笨拙\tabularnewline
\bottomrule
\end{tabular}}
\end{table}





\begin{table}[h!]
\caption{认知功能的7*盲点(Trickster)表现:无意识的黑洞。
我们不仅不擅长此功能,而且往往会通过贬低它来保护自尊。表现形式:根本看不见其存在的必要性,将其视为多余、虚伪或荒谬的干扰。}
    \label{tab:cog-function-trickster}
\centering
\resizebox{\textwidth}{!}{
\begin{tabular}{cccc}
\toprule
盲点功能 & 涉及类型 & 盲区心态/潜台词 (The Dismissal) & 盲点具象表现\tabularnewline
\midrule
\multicolumn{4}{c}{对人际与道德的盲视}\tabularnewline
\midrule
Fe & INTJ, ISTJ & “虚伪的客套有什么用?讲重点。” & 视社交礼仪为操纵,善意说真话却成冒犯,完全读不懂空气\tabularnewline
Fi & ENTP, ESTP & “这也太矫情了,客观一点行不行?” & 情感逻辑化,不知道自己真实感受,无意间把别人的价值观踩得粉碎\tabularnewline
\midrule
\multicolumn{4}{c}{对现实与身体的盲视}\tabularnewline
\midrule
Se & INTP, INFP & “这重要吗?我没看见/没听见。” & 物理笨拙(平地摔),极其忽视环境细节,活在脑洞里完全缺乏行动力\tabularnewline
Si & ENTJ, ENFJ & “别提过去,我只要未来!” & 严重透支身体而不自知,极度厌恶怀旧和重复,记不住生活琐事\tabularnewline
\midrule
\multicolumn{4}{c}{对联想与愿景的盲视}\tabularnewline
\midrule
Ne & ISTP, ISFP & “你到底想说什么?别扯那些没用的。” & 极度字面化,理解不了暗示和隐喻,讨厌猜测,一条路走到黑\tabularnewline
Ni & ESTJ, ESFJ & “别想太多,先做再说,这不切实际。” & 短期主义,只要确定的流程而不要愿景,瞎操心,对未知充满恐惧\tabularnewline
\midrule
\multicolumn{4}{c}{对逻辑与效率的盲视}\tabularnewline
\midrule
Te & INFJ, ISFJ & “你这人怎么这么冷血/功利?” & 无法理清外部资源,难以做客观决策,为了维护人际或信仰而抗拒数据\tabularnewline
Ti & ENFP, ESFP & “太较真了没意思,差不多就行了。” & 逻辑疏松,讨厌定义和辩论,经常自我矛盾却不自知,只关注“好不好”\tabularnewline
\bottomrule
\end{tabular}}
\end{table}


\begin{table}

\caption{认知功能的8*恶魔(Demon)表现: 彻底崩溃时的毁灭打击。这是人格结构中最深层,最黑暗,也是最原始的部分。触发条件:自尊彻底崩溃,核心界限被严重侵犯,或者感受到极度绝望。表现形式:毁灭一切,不在乎输赢,只想让一切归零
-- “既然你毁了我的世界,那我就把整个宇宙都炸了陪葬”。转化潜力:如果能整合,它也能带来涅槃重生的力量(The Daemonic
vs. The Demonic)。}
    \label{tab:cog-function-demon}
    \centering
\resizebox{\textwidth}{!}{
\begin{tabular}{cccc}
\toprule
恶魔功能 & 涉及类型 & 核心心态 & 毁灭性表现(The Destruction)\tabularnewline
\midrule
\multicolumn{4}{c}{感知型恶魔}\tabularnewline
\midrule
Si Demon & INTJ, INFJ & 彻底抹除 | Erasure & 记忆清洗与断连(The Door Slam),肉体自毁倾向\tabularnewline
Se Demon & ENTP, ENFP & 现实崩塌 | Burn it down & 物理毁灭与虚无,极度慵懒或狂躁\tabularnewline
Ni Demon & ISTJ, ISFJ & 永恒绝望 | Hopelessness & 恐怖末日预言家,否定别人的梦想\tabularnewline
Ne Demon & ISTP, ISFP & 恶意扭曲 | Gaslighting & 被害妄想与构陷,混乱制造者\tabularnewline
\midrule
\multicolumn{4}{c}{判断型恶魔}\tabularnewline
\midrule
Ti Demon & INFP, ISFP & 诛心真话 | Cold Truth & 最恶毒的逻辑穿刺,摧毁对方自尊\tabularnewline
Te Demon & INTP, ISTP & 系统毁灭 | System Crash & 冷酷的暴君,破坏性执行,无视人性\tabularnewline
Fi Demon & ENTP, ESTP & 道德虚无 | Nihilism & 反社会倾向,情感报复\tabularnewline
Fe Demon & ENTJ, ESTJ & 名誉屠杀 | Character Assassination & 情感操纵与社会死亡,情感爆发\tabularnewline
\bottomrule
\end{tabular}}
\end{table}


\begin{table}

\caption{认知功能的8*恶魔(Demon)整合:涅槃重生后的救赎之力(The Daemonic)。当恶魔功能被意识整合,它不再是毁灭者,而变成了通往完整人格的守门人。这种力量通常带有一种“非个人”的、超越小我的神性或宿命感。表现形式:大智若愚、绝地反击、或是对他人的深刻救赎。}
    \label{tab:cog-function-demon-redemption}
    \centering

\resizebox{\textwidth}{!}{
\begin{tabular}{cccc}
\toprule
恶魔功能 & 涉及类型 & 转化核心 & 整合后表现(The Redemption)\tabularnewline
\midrule
\multicolumn{4}{c}{感知型恶魔}\tabularnewline
\midrule
Si Demon & INTJ, INFJ & 历史和解 | Reconciliation & 成为传统的守护者,从过往创伤中提炼史诗般的意义,极致的身体掌控\tabularnewline
Se Demon & ENTP, ENFP & 绝对显化 | Manifestation & 将抽象疯狂的灵感在现实中完美落地,极具感染力的舞台表现,活在当下\tabularnewline
Ni Demon & ISTJ, ISFJ & 先知智慧 | Visionary & 突破经验主义的桎梏,在危机时刻涌现出精准的未来预判,指引方向\tabularnewline
Ne Demon & ISTP, ISFP & 万物互联 | Interconnection & 极其通透的开放心态,理解复杂的混沌系统,成为打破僵局的创新者\tabularnewline
\midrule
\multicolumn{4}{c}{判断型恶魔}\tabularnewline
\midrule
Ti Demon & INFP, ISFP & 结构真理 | Structural Truth & 能够用最客观的逻辑捍卫价值观,不带情绪地通过理性解决最棘手的问题\tabularnewline
Te Demon & INTP, ISTP & 建设秩序 | Constructive Order & 极具执行力的系统构建者,在混乱中建立坚不可摧的帝国,为了大局的高效\tabularnewline
Fi Demon & ENTP, ESTP & 慈悲圣徒 | Universal Love & 浪子回头般的真诚,对他人的苦难产生深刻共情,为了信仰不惜一切\tabularnewline
Fe Demon & ENTJ, ESTJ & 仁慈领袖 | Benevolent Ruler & 放下权力的傲慢,展现出真正的人性光辉,通过情感连接凝聚人心\tabularnewline
\bottomrule
\end{tabular}}
\end{table}










\begin{table}[h!]
\caption{认知功能的防御机制/长期内耗循环(Loop)。不同于瞬间压力的Grip,Loop是一种长期的、病态的舒适区,个体切断了辅助功能(第二功能)的平衡作用,只在第一和第三功能之间死循环。}
    \label{tab:cog-function-loop}
    \centering
    \resizebox{\textwidth}{!}{
\begin{tabular}{cccc}
\toprule
Loop组合 & 状态名称 & 涉及类型 & 典型表现 (Manifestation)\tabularnewline
\midrule
\multicolumn{4}{c}{内向Loop:切断外部互动 $\rightarrow$ 陷入大脑死循环 $\rightarrow$ 拒绝现实检验}\tabularnewline
\midrule
\multirow{2}{*}{Ni - Fi} & \multirow{2}{*}{偏执内耗 (Paranoid)} & INTJ & 沉溺消极愿景,觉得自己是受害者,全世界都在针对自己,无行动验证\tabularnewline
 & & ISFP & 陷入深刻的负面自我定义,确信无人能理解自己,自我孤立\tabularnewline
\midrule
\multirow{2}{*}{Ti - Si} & \multirow{2}{*}{固步自封 (Stuck)} & INTP & 反复分析过去错误,抗拒新信息,甚至忽略饮食卫生等基本生存需求\tabularnewline
 & & ISFJ & 翻旧账大师,逻辑证明自己注定失败,拒绝任何新尝试\tabularnewline
\midrule
\multirow{2}{*}{Ni - Ti} & \multirow{2}{*}{虚无主义 (Nihilistic)} & INFJ & 极度冷漠愤世嫉俗,切断同情心,构建宏大理论证明“人类没救了”\tabularnewline
 & & ISTP & 对未来产生阴谋论,逻辑自洽地认为别人甚至宇宙都在算计自己\tabularnewline
\midrule
\multirow{2}{*}{Fi - Si} & \multirow{2}{*}{受害循环 (Resentful)} & INFP & 强迫性重温过去创伤,认为未来只会重复痛苦,丧失所有希望\tabularnewline
 & & ISTJ & 变得情绪化且敏感,过度解读过去的细节,赋予其沉重的负面色彩\tabularnewline
\midrule
\multicolumn{4}{c}{外向Loop:切断内心联系 $\rightarrow$ 疯狂外部活动 $\rightarrow$ 逃避深度思考}\tabularnewline
\midrule
\multirow{2}{*}{Te - Se} & \multirow{2}{*}{鲁莽暴君 (Bulldozer)} & ENTJ & 为行动而行动,停不下来,通过高强度的感官/工作麻痹思考,独断专行\tabularnewline
 & & ESFP & 极度强势急功近利,不仅自己享乐,还强迫别人服从自己的“效率”\tabularnewline
\midrule
\multirow{2}{*}{Ne - Fe} & \multirow{2}{*}{表演人格 (Manipulative)} & ENTP & 极度渴望关注,失去逻辑锚点,为获得反应故意挑衅或成为空洞梗王\tabularnewline
 & & ESFJ & 脑洞大开地焦虑他人看法,变得八卦、戏剧化,为表面和谐撒谎\tabularnewline
\midrule
\multirow{2}{*}{Fe - Se} & \multirow{2}{*}{浅薄迎合 (Superficial)} & ENFJ & 沉迷当下感官与评价,冲动消费,过度社交,忘记长远目标\tabularnewline
 & & ESTP & 虚伪的社交变色龙,为了群体认可过度表现情感,在不同圈子演戏\tabularnewline
\midrule
\multirow{2}{*}{Ne - Te} & \multirow{2}{*}{焦虑忙碌 (Burnout)} & ENFP & 开启一百个计划却无一完成,强迫性快速执行,无法容忍情感反思\tabularnewline
 & & ESTJ & 异想天开却独断专行,提出不切实际的新规并强迫大家执行,抛弃经验\tabularnewline
\bottomrule
\end{tabular}}
\end{table}



\begin{table}[h!]
\caption{长期内耗循环(Loop)的解药。核心逻辑:Loop 是因为忽略了辅助功能(家长),导致第一与第三功能恶性循环。解药是\textbf{强制调用第二功能}。}
    \label{tab:cog-function-loop-remedy}
    \centering
    \resizebox{\textwidth}{!}{
\begin{tabular}{cccc}
\toprule
解药 (辅助功能) & 涉及 Loop & 核心逻辑 (Logic) & 行动指南 (Actionable Steps)\tabularnewline
\midrule
\multicolumn{4}{c}{内向 Loop 解药:向外连接,破除自我封闭 (Externalize)}\tabularnewline
\midrule
\multirow{2}{*}{Te (外倾思考)} & Ni-Fi (INTJ) & \textbf{产出与秩序} & \multirow{2}{*}{列出清单,整理物理环境,追求“小赢”,将担忧转化为待办}\tabularnewline
 & Si-Fi (ISTJ) & 停止感受,开始执行 & \tabularnewline
\midrule
\multirow{2}{*}{Fe (外倾情感)} & Ni-Ti (INFJ) & \textbf{表达与连接} & \multirow{2}{*}{找人倾诉,进行辅导/帮助他人,寻求外部的情感确认}\tabularnewline
 & Si-Ti (ISFJ) & 停止自剖,关注他人 & \tabularnewline
\midrule
\multirow{2}{*}{Se (外倾感觉)} & Ti-Ni (ISTP) & \textbf{行动与感官} & \multirow{2}{*}{高强度运动,手工制造,大扫除,接触自然,强迫回到当下}\tabularnewline
 & Fi-Ni (ISFP) & 停止空想,物理接触 & \tabularnewline
\midrule
\multirow{2}{*}{Ne (外倾直觉)} & Ti-Si (INTP) & \textbf{探索与发散} & \multirow{2}{*}{物理换环境,摄入随机信息(书/剧),头脑风暴,尝试新路线}\tabularnewline
 & Fi-Si (INFP) & 停止反刍,寻找新意 & \tabularnewline
\midrule
\multicolumn{4}{c}{外向 Loop 解药:内向沉淀,停下来检查内心 (Internalize)}\tabularnewline
\midrule
\multirow{2}{*}{Ni (内倾直觉)} & Te-Se (ENTJ) & \textbf{愿景与暂停} & \multirow{2}{*}{强制独处,断网冥想,推演长远后果,追问做事的根本意义}\tabularnewline
 & Fe-Se (ENFJ) & 停止忙碌,寻找本质 & \tabularnewline
\midrule
\multirow{2}{*}{Si (内倾感觉)} & Te-Ne (ESTJ) & \textbf{复盘与稳健} & \multirow{2}{*}{查阅历史先例,关注身体感受,做可行性检查,休息与维护}\tabularnewline
 & Fe-Ne (ESFJ) & 停止折腾,回归经验 & \tabularnewline
\midrule
\multirow{2}{*}{Fi (内倾情感)} & Ne-Te (ENFP) & \textbf{价值与本心} & \multirow{2}{*}{日记反思,道德审查,学会说“不”,确认是否符合真实喜好}\tabularnewline
 & Se-Te (ESFP) & 停止取悦,自问本心 & \tabularnewline
\midrule
\multirow{2}{*}{Ti (内倾思考)} & Ne-Fe (ENTP) & \textbf{逻辑与真理} & \multirow{2}{*}{独立逻辑分析,精确定义概念,消除模糊,不论人情只论对错}\tabularnewline
 & Se-Fe (ESTP) & 停止表演,回归事实 & \tabularnewline
\bottomrule
\end{tabular}}
\end{table}













\end{document}
