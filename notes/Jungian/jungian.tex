\documentclass[UTF8,letter,10pt]{ctexart}
\usepackage{geometry}
\geometry{left=2.5cm, right=2.5cm, top=2.5cm, bottom=2.5cm}
\setmainfont{Times New Roman}
\usepackage{booktabs}
\usepackage{colortbl}
\usepackage{xcolor}
\usepackage{microtype}
\usepackage{multirow}
\usepackage{makecell}
\usepackage{graphicx}
\usepackage{float}

\definecolor{intjpurple}{HTML}{88619A}
%\definecolor{intjpurple}{HTML}{4A148C}


\begin{document}

\begin{table}[h!]
    \caption{MBTI - 荣格八维映射表}
    \centering
    \begin{tabular}{cccc}
        \toprule
        \multicolumn{4}{c}{\textbf{NT -- 分析师 Analysts}} \\
        \cellcolor{intjpurple}\color{white} INTJ (Ni-Te-Fi-Se) & INTP (Ti-Ne-Si-Fe) & ENTJ (Te-Ni-Se-Fi) & ENTP (Ne-Ti-Fe-Si) \\
        \midrule
        \multicolumn{4}{c}{\textbf{NF -- 外交官 Diplomats}} \\
        INFJ (Ni-Fe-Ti-Se) & INFP (Fi-Ne-Si-Te) & ENFJ (Fe-Ni-Se-Ti) & ENFP (Ne-Fi-Te-Si) \\
        \midrule
        \multicolumn{4}{c}{\textbf{SJ -- 守护者 Sentinels}} \\
        ISTJ (Si-Te-Fi-Ne) & ISFJ (Si-Fe-Ti-Ne) & ESTJ (Te-Si-Ne-Fi) & ESFJ (Fe-Si-Ne-Ti) \\
        \midrule
        \multicolumn{4}{c}{\textbf{SP -- 探索者 Explorers}} \\
        ISTP (Ti-Se-Ni-Fe) & ISFP (Fi-Se-Ni-Te) & ESTP (Se-Ti-Fe-Ni) & ESFP (Se-Fi-Te-Ni) \\
        \bottomrule
    \end{tabular}
\end{table}


\begin{table}[h!]
    \caption{认知功能的角色表}
    \centering
    \begin{tabular}{cccc}
    \toprule
    序号 & 位置 & 角色名称 & 典型表现\tabularnewline
    \midrule
        1 & 主导 (Dom) & 英雄(Hero) & 核心力量\tabularnewline
        2 & 辅助 (Auxiliary) & 家长(Parent) & 辅助工具\tabularnewline
        3 & 第三 (Tert) & 孩子(Child) & 舒适区\tabularnewline
        4 & 劣势 (Inferior) & 阿尼玛(Anima) & 压力源泉/理想\tabularnewline
    \midrule
    5 & 影子 & 反向作用(Opposing) & 质疑者\tabularnewline
    6 & 影子 & 批判家长(Critical) & 挑刺者\tabularnewline
    \rowcolor{gray!17}7 & 影子 & 盲点(Trickster) & 真正盲区\tabularnewline
    8 & 影子 & 恶魔(Demon) & 毁灭者\tabularnewline
    \bottomrule
\end{tabular}
\end{table}


\begin{table}[h!]
    \caption{认知功能的正常表现}
    \centering
\begin{tabular}{cccc}
\toprule
简称 & 全称 & 核心词 & 正常表现\tabularnewline
\midrule
Se & Extraverted Sensing & 体验,当下,刺激 & 活在当下,直接行动,现实主义\tabularnewline
Si & Introverted Sensing & 经验,细节,稳健 & 参照经验,细节敏感,追求守序\tabularnewline
Ne & Extraverted Intuition & 发散,脑洞,可能 & 跳跃思维,探索新奇,抽象关联\tabularnewline
Ni & Introverted Intuition & 收敛,洞察,本质 & 预测未来,现象本质,单一愿景\tabularnewline
Te & Extraverted Thinking & 效率,执行,调配 & 结果导向,逻辑外化,强势直接\tabularnewline
Ti & Introverted Thinking & 逻辑,解构,精准 & 追求真理,刨根问底,无视权威\tabularnewline
Fe & Extraverted Feeling & 和谐,人际,共情 & 情绪传染,维护关系,社交礼仪\tabularnewline
Fi & Introverted Feeling & 本心,真实,价值 & 忠于自我,情感内化,排斥虚伪\tabularnewline
\bottomrule
\end{tabular}

\end{table}


\begin{table}
\caption{认知功能的6*批判家长(Critical Parents)表现: 如何打压别人和自己}
    \resizebox{\textwidth}{!}{
\begin{tabular}{ccccc}
\toprule
批判功能 & 涉及类型 & 核心台词 & 对外打压 & 对内自毁\tabularnewline
\midrule
\multicolumn{5}{c}{理性审判组:Judgement}\tabularnewline
\midrule
Ti Critical 批判逻辑 & INTJ, ISTJ & 你真蠢 & 智力羞辱,冷漠指出他人逻辑漏洞 & 自我怀疑,质疑自己不够聪明\tabularnewline
Fi Critical 批判道德 & INFJ, ISFJ & 你真坏 & 道德审判,指责自私虚伪冷血人品 & 自我羞耻,觉得自己是伪君子\tabularnewline
Te Critical 批判能力 & ENTP, ESTP & 你没用 & 能力否定,嘲笑笨手笨脚一事无成 & 成就焦虑,成功仍觉一无是处\tabularnewline
Fe Critical 批判社交 & ENFP, ESFP & 没人理你 & 社交孤立,攻击对方情商或者礼仪 & 归属崩塌,总觉自己格格不入\tabularnewline
\midrule
\multicolumn{5}{c}{感知审判组:Perception}\tabularnewline
\midrule
Ni Critical 批判远见 & INTP, INFP & 没希望 & 未来诅咒,否定愿景断言注定失败 & 虚无主义,觉得做什么都没意义\tabularnewline
Si Critical 批判经验 & ISTP, ISFP & 你真弱 & 翻旧帐,指责不可靠不长记性/否定毅力 & 创伤反刍,不断回忆过去的痛苦\tabularnewline
Ne Critical 批判意图 & ENTJ, ENFJ & 你想害我 & 动机阴谋论,怀疑忠诚指责对方鲁莽 & 选择障碍,看到无数会感到受困\tabularnewline
Se Critical 批判现实 & ESTJ, ESFJ & 真难看 & 外貌/品味攻击,苛刻挑剔对方衣着审美 & 容貌/现实焦虑,觉得自己丑陋笨拙\tabularnewline
\bottomrule
\end{tabular}}
\end{table}





\begin{table}[h!]
    \caption{认知功能的7*盲点(Trickster)表现}
    \centering
\begin{tabular}{ccc}
\toprule
盲点功能 & 核心特征 & 盲点表现\tabularnewline
\midrule
Fe (INTJ, ISTJ) & 社交迟钝 & 视社交礼仪为虚伪和操纵,善意说真话说真话却成冒犯,情感处理生硬且滞后\tabularnewline
Fi (ENTP, ESTP) & 道德相对主义 & 不知道自己感受,情感逻辑化,无意间践踏别人价值观\tabularnewline
Se (INTP, INFP) & 现实锻联 & 物理笨拙平地摔,忽视环境,活在云端缺乏行动力\tabularnewline
Si (ENTJ, ENFJ) & 肉体忽视 & 不仅是记性差,透支身体,厌恶怀旧\tabularnewline
Ne (ISTP, ISFP) & 缺乏联想 & 极度字面化,不理解暗示,讨厌猜测,一条路走到黑\tabularnewline
Ni (ESTJ, ESFJ) & 缺乏远见 & 短期主义,只要确定的而不要直觉和愿景,瞎操心\tabularnewline
Te (INFJ, ISFJ) & 逻辑混沌 & 无法通过思考做决定,资源管理困难,维护人际或信仰甚至反智\tabularnewline
Ti (ENFP, ESFP) & 逻辑疏松 & 差不多就行,讨厌理论辩论,逻辑自相矛盾却不自知\tabularnewline
\bottomrule
\end{tabular}

\end{table}

%\section{认知功能的恶魔(Demon)表现: 彻底崩溃时的毁灭打击}












\begin{table}[h!]
\caption{认知功能的防御机制/长期内耗循环(Loop)}
    \centering
\begin{tabular}{cccc}
\toprule 
Loop名称 & 状态描述 & 涉及类型 & 典型表现\tabularnewline
\midrule
\multicolumn{4}{c}{内向Loop共同特征:切断外部世界互动,陷入大脑死循环,拒绝现实检验}\tabularnewline
\midrule
\multirow{2}{*}{Ni - Fi} & \multirow{2}{*}{偏执内耗 (Paranoid)} & INTJ & \makecell{觉得自己是受害者,全世界都在针对自己;\\ 沉溺消极愿景,却没有实际行动改变或验证}\tabularnewline
 &  & ISFP & \makecell{沉浸在某种深刻消极的自我定义中,觉得没人理解自己}\tabularnewline
 \midrule
\multirow{2}{*}{Ti - Si} & \multirow{2}{*}{固步自封 (Stuck)} & INTP & \makecell{重复分析过去的错误细节;\\ 抗拒新信息变得守旧,甚至忽略饮食卫生}\tabularnewline
 &  & ISFJ & \makecell{过度分析过去尴尬瞬间,逻辑证明自己注定失败,\\ 拒绝尝试任何新事物}\tabularnewline
 \midrule
\multirow{2}{*}{Ni -Ti} & \multirow{2}{*}{虚无主义 (Nihilistic)} & INFJ & \makecell{极其冷漠和愤世嫉俗;脑中构建理论证明人类药丸;\\ 切断对他人的同情}\tabularnewline
 &  & ISTP & \makecell{对未来产生阴谋论,并觉得逻辑自洽;\\ 变得生性多疑难以接近}\tabularnewline
 \midrule
\multirow{2}{*}{Fi - Si} & \multirow{2}{*}{受害循环 (Resentful)} & INFP & \makecell{强迫性地重温过去创伤和尴尬;\\ 认为未来只会重复过去的痛苦,丧失对可能性的希望}\tabularnewline
 &  & ISTJ & \makecell{变得极度情绪化和敏感;\\ 沉溺过去细节并赋予负面感情色彩}\tabularnewline
 \midrule
\multicolumn{4}{c}{外向Loop共同特征:切断内心世界联系,通过疯狂的外部活动逃避思考和反省}\tabularnewline
\midrule
\multirow{2}{*}{Te - Se} & \multirow{2}{*}{鲁莽暴君 (Bulldozer)} & ENTJ & \makecell{为了行动而行动,停不下来;\\ 独断专行,只要结果不要过程,甚至通过Se来麻痹思考}\tabularnewline
 &  & ESFP & \makecell{极度强势和急功近利;试图控制一切;\\ 不仅感受当下,而且强迫别人服从自己的效率}\tabularnewline
\midrule 
\multirow{2}{*}{Ne - Fe} & \multirow{2}{*}{表演人格 (Manipulative)} & ENTP & \makecell{极度渴望关注,为了获得反应故意挑衅或者取悦他人;\\ 失去内在逻辑锚点,成为空洞的梗王}\tabularnewline
 &  & ESFJ & \makecell{南洞大开的担心别人怎么看自己;变得八卦,戏剧化;\\ 为了表面和谐而撒谎,完全失去原则}\tabularnewline
\midrule 
\multirow{2}{*}{Fe - Se} & \multirow{2}{*}{浅薄迎合 (Superficial)} & ENFJ & \makecell{极度在乎当下感官刺激和他人评价;\\ 冲动消费,过度社交,忘记原本长远目标}\tabularnewline
 &  & ESTP & \makecell{为了群体认可过度表现情感;\\ 变得虚伪,为了显得合群在各种圈子里一个接一个的演}\tabularnewline
\midrule 
\multirow{2}{*}{Ne - Te} & \multirow{2}{*}{焦虑忙碌 (Burnout)} & ENFP & \makecell{没有目标的同时开启一百个计划,并强迫性的快速执行;\\ 变得专横,暴躁,无法容忍任何情感反思}\tabularnewline
 &  & ESTJ & \makecell{变得异想天开却又独断专行;\\ 提出不切实际的新规并强迫大家执行,抛弃过去经验}\tabularnewline
 \bottomrule
\end{tabular}
\end{table}



\begin{table}[h!]
\caption{长期内耗循环的解药。核心逻辑:通过强制调用辅助功能(第二)来打破循环。}
    \centering
\resizebox{\textwidth}{!}{
\begin{tabular}{cccc}
\toprule
解药(强制调用第二) & 涉及Loop & 核心逻辑 & 行动指南\tabularnewline
\midrule
\multicolumn{4}{c}{内向Loop解药:向外连接,破除自我封闭}\tabularnewline
\midrule
\multirow{2}{*}{Te} & Ni-Fi (INTJ) & 产出与秩序 & \multirow{2}{*}{列清单(担忧变成待办事项),整理环境,追求小赢}\tabularnewline
 & Si-Fi (ISTJ) & 停止感受,开始执行 & \tabularnewline
\midrule
\multirow{2}{*}{Fe} & Ni-Ti (INFJ) & 表达与连接 & \multirow{2}{*}{倾诉,帮他人的忙,向他人确认感受}\tabularnewline
 & Si-Ti (ISFJ) & 停止自我分析,关注他人感受 & \tabularnewline
\midrule
\multirow{2}{*}{Se} & Ti-Ni (ISTP) & 行动与感官 & \multirow{2}{*}{高强度运动,立即可见的手工和操作,感官刺激}\tabularnewline
 & Fi-Ni (ISFP) & 停止空想未来,强迫回到当下 & \tabularnewline
\midrule
\multirow{2}{*}{Ne} & Ti-Si (INTP) & 探索与发散 & \multirow{2}{*}{换个环境,随机输入(书籍/电影),头脑风暴}\tabularnewline
 & INFP (Fi-Si) & 停止反刍过去,寻找新的可能 & \tabularnewline
\midrule
\multicolumn{4}{c}{外向Loop解药:内向沉淀,停下来检查内心和自省}\tabularnewline
\midrule
\multirow{2}{*}{Ni} & Te-Se (ENTJ) & 愿景与暂停 & \multirow{2}{*}{强制独处,推演后果,寻找本质}\tabularnewline
 & Fe-Se (ENFJ) & 停止忙碌,思考长远意义和后果 & \tabularnewline
\midrule
\multirow{2}{*}{Si} & Te-Ne (ESTJ) & 复盘与稳健 & \multirow{2}{*}{查阅先例,关注身体,落地检查(推敲可行性和细节)}\tabularnewline
 & Fe-Ne (ESFJ) & 停止折腾新花样,回归经验和教训 & \tabularnewline
\midrule
\multirow{2}{*}{Fi} & Ne-Te (ENFP) & 价值与本心 & \multirow{2}{*}{日记反思,道德审查,学会拒绝来保护价值观}\tabularnewline
 & Se-Te (ESFP) & 停止取悦和效率,自问内心真实喜好 & \tabularnewline
\midrule
\multirow{2}{*}{Ti} & Ne-Fe (ENTP) & 逻辑与真理 & \multirow{2}{*}{逻辑自洽,独立分析,精确定义概念,消除模糊}\tabularnewline
 & Se-Fe (ESTP) & 停止表演和操纵,回归客观事实 & \tabularnewline
\bottomrule
\end{tabular}}
\end{table}










\begin{table}[h!]
\caption{认知功能的极端压力表现。
    注意:功能发育不成熟,或者日常的防御机制(loop)不同于极端压力/抓取状态(grip)}
    \centering
\resizebox{\textwidth}{!}{
\begin{tabular}{ccc}
\toprule
劣势功能 & 触发点 & 极端压力爆发/抓取表现\tabularnewline
\midrule
Se (INTJ, INFJ) & 脑力过度,理论受挫,现实失控 & 感官放纵,强迫行为,鲁莽冲动,暴饮暴食,报复性熬夜\tabularnewline
Si (ENTP, ENFP) & 细节淹没,感到束缚,身体不适 & 怀疑绝症,抑郁退缩,钻牛角尖\tabularnewline
Ne (ISTJ, ISFJ) & 生活未知,计划打乱,快速应变 & 灾难思维,阴谋论,胡乱联系\tabularnewline
Ni (ESTP, ESFP) & 被迫承诺,理论困扰,未来无望 & 宿命论,黑暗妄想,错误解读\tabularnewline
Te (INFP, ISFP) & 价值践踏,自觉无能,强权压迫 & 暴君模式,急功近利,归咎他人\tabularnewline
Ti (ENFJ, ESFJ) & 人际破裂,不被尊重,逻辑混乱 & 冷酷逻辑,过度分析,自我贬低\tabularnewline
Fe (INTP, ISTP) & 被忽视,逻辑无效,情感冲突 & 情绪失控,受害心态,过度情感解读\tabularnewline
Fi (ENTJ, ESTJ) & 权威被挑战,计划失败,感到孤独 & 情感泛滥,被动攻击,甚至抑郁\tabularnewline
\bottomrule
\end{tabular}}
\end{table}


\begin{table}[h!]
\caption{极端压力/抓取状态的解药。Grip状态意味着主导功能易经枯竭,此时的解药是安抚劣势功能,而千万不要再用主导功能对抗劣势功能。}
    \resizebox{\textwidth}{!}{
\begin{tabular}{ccccc}
\toprule
劣势功能 & 解药 & 涉及类型 & 压力源泉 & 执行建议\tabularnewline
\midrule
\multicolumn{5}{c}{劣势感知功能:停止对意义和可能性的过度挖掘,回归基础生理和现实}\tabularnewline
\midrule
\multirow{2}{*}{Se} & 低压感官疗法 & INTJ & 脑力透支 & \multirow{2}{*}{高质量睡眠,轻度运动/自然,摄影}\tabularnewline
 & 适度,无目的的感官享受 & INFJ & 大量细节干扰 & \tabularnewline
\midrule
\multirow{2}{*}{Si} & 熟悉感与滋养 & ENTP & 感到束缚 & \multirow{2}{*}{吃舒适的食物,温水澡,简化待办事项}\tabularnewline
 & 停止探索新世界,回到安全屋 & ENFP & 细节淹没 & \tabularnewline
\midrule
\multicolumn{5}{c}{劣势直觉功能:停止当下的过度反应或控制,找回一点未来希望或者新视角}\tabularnewline
\midrule
\multirow{2}{*}{Ni} & 放过明天 & ESTP & 被迫承诺 & \multirow{2}{*}{借用他人大脑,推迟决定,转移注意力}\tabularnewline
 & 接受无法解决未来问题,寻求他人愿景支持 & ESFP & 未来受阻 & \tabularnewline
\midrule
\multirow{2}{*}{Ne} & 现实检验 & ISTJ & 计划被打乱 & \multirow{2}{*}{最坏也就这样,找人分担,主动做出微小改变}\tabularnewline
 & 用过去确凿的经验对抗虚幻想象 & ISFJ & 未知的环境 & \tabularnewline
\midrule
\multicolumn{5}{c}{劣势思考功能:停止对情感和人际的过度纠结,通过客观小任务找回控制感}\tabularnewline
\midrule
\multirow{2}{*}{Te} & 小赢策略 & INFP & 价值观被践踏 & \multirow{2}{*}{整理一个抽屉,划掉任务单,承认无力}\tabularnewline
 & 放下宏大目标,做一件必定成功的简单事 & ISFP & 感到无能 & \tabularnewline
\midrule
\multirow{2}{*}{Ti} & 逻辑隔离 & ENFJ & 人际冲突 & \multirow{2}{*}{独处充电,解谜游戏,停止读心}\tabularnewline
 & 暂时切断情感雷达,做不涉及人的智力活动 & ESFJ & 缺乏尊重 & \tabularnewline
\midrule
\multicolumn{5}{c}{劣势情感功能:停止高强度逻辑分析,承认自己是有血有肉的人,接受情绪宣泄}\tabularnewline
\midrule
\multirow{2}{*}{Fi} & 允许脆弱 & ENTJ & 权威受挫 & \multirow{2}{*}{私密环境宣泄,暂停工作,寻找盟友}\tabularnewline
 & 不要试图分析和解决情绪,仅仅是感受它 & ESTJ & 感到孤独 & \tabularnewline
\midrule
\multirow{2}{*}{Fe} & 无压陪伴 & INTP & 逻辑无效 & \multirow{2}{*}{陪伴但各玩各的,书面沟通,身体接触}\tabularnewline
 & 不是彻底隔离或高压社交,而是轻松共处 & ISTP & 被忽视 & \tabularnewline
\bottomrule
\end{tabular}}
\end{table}




\end{document}
