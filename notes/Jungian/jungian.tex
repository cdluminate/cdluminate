\documentclass[UTF8,letter,10pt]{ctexart}
\usepackage{geometry}
\geometry{left=2.5cm, right=2.5cm, top=2.5cm, bottom=2.5cm}
\setmainfont{Times New Roman}
\usepackage{booktabs}
\usepackage{colortbl}
\usepackage{xcolor}
\usepackage{microtype}

\definecolor{intjpurple}{HTML}{88619A}
%\definecolor{intjpurple}{HTML}{4A148C}


\begin{document}

\section{MBTI/荣格八维认知功能表}

\begin{table}[h!]
    \centering
    \begin{tabular}{cccc}
        \toprule
        \multicolumn{4}{c}{\textbf{NT -- 分析师 Analysts}} \\
        \cellcolor{intjpurple}\color{white} INTJ (Ni-Te-Fi-Se) & INTP (Ti-Ne-Si-Fe) & ENTJ (Te-Ni-Se-Fi) & ENTP (Ne-Ti-Fe-Si) \\
        \midrule
        \multicolumn{4}{c}{\textbf{NF -- 外交官 Diplomats}} \\
        INFJ (Ni-Fe-Ti-Se) & INFP (Fi-Ne-Si-Te) & ENFJ (Fe-Ni-Se-Ti) & ENFP (Ne-Fi-Te-Si) \\
        \midrule
        \multicolumn{4}{c}{\textbf{SJ -- 守护者 Sentinels}} \\
        ISTJ (Si-Te-Fi-Ne) & ISFJ (Si-Fe-Ti-Ne) & ESTJ (Te-Si-Ne-Fi) & ESFJ (Fe-Si-Ne-Ti) \\
        \midrule
        \multicolumn{4}{c}{\textbf{SP -- 探索者 Explorers}} \\
        ISTP (Ti-Se-Ni-Fe) & ISFP (Fi-Se-Ni-Te) & ESTP (Se-Ti-Fe-Ni) & ESFP (Se-Fi-Te-Ni) \\
        \bottomrule
    \end{tabular}
\end{table}

\section{认知功能的角色表}

\begin{table}[h!]
    \centering
    \begin{tabular}{cccc}
    \toprule
    序号 & 位置 & 角色名称 & 典型表现\tabularnewline
    \midrule
        1 & 主导 (Dom) & 英雄(Hero) & 核心力量\tabularnewline
        2 & 辅助 & 家长(Parent) & 辅助工具\tabularnewline
        3 & 第三 (Tert) & 孩子(Child) & 舒适区\tabularnewline
        4 & 劣势 (Inferior) & 阿尼玛(Anima) & 压力源泉/理想\tabularnewline
    \midrule
    5 & 影子 & 反向作用(Opposing) & 质疑者\tabularnewline
    6 & 影子 & 批判家长(Critical) & 挑刺者\tabularnewline
    \rowcolor{gray!17}7 & 影子 & 盲点(Trickster) & 真正盲区\tabularnewline
    8 & 影子 & 恶魔(Demon) & 毁灭者\tabularnewline
    \bottomrule
\end{tabular}
\end{table}

\section{认知功能的正常表现}

\begin{table}[h!]
    \centering
\begin{tabular}{cccc}
\toprule
简称 & 全称 & 核心词 & 正常表现\tabularnewline
\midrule
Se & Extraverted Sensing & 体验,当下,刺激 & 活在当下,直接行动,现实主义\tabularnewline
Si & Introverted Sensing & 经验,细节,稳健 & 参照经验,细节敏感,追求守序\tabularnewline
Ne & Extraverted Intuition & 发散,脑洞,可能 & 跳跃思维,探索新奇,抽象关联\tabularnewline
Ni & Introverted Intuition & 收敛,洞察,本质 & 预测未来,现象本质,单一愿景\tabularnewline
Te & Extraverted Thinking & 效率,执行,调配 & 结果导向,逻辑外化,强势直接\tabularnewline
Ti & Introverted Thinking & 逻辑,解构,精准 & 追求真理,刨根问底,无视权威\tabularnewline
Fe & Extraverted Feeling & 和谐,人际,共情 & 情绪传染,维护关系,社交礼仪\tabularnewline
Fi & Introverted Feeling & 本心,真实,价值 & 忠于自我,情感内化,排斥虚伪\tabularnewline
\bottomrule
\end{tabular}

\end{table}

\section{认知功能的极端压力表现}

注意:功能发育不成熟,或者日常的防御机制(loop)不同于极端压力/抓取状态(grip)

\begin{table}[h!]
    \centering
\begin{tabular}{ccc}
\toprule
劣势功能 & 触发点 & 抓取表现\tabularnewline
\midrule
Se (INTJ, INFJ) & 脑力过度,理论受挫,现实失控 & 感官放纵,强迫行为,鲁莽冲动\tabularnewline
Si (ENTP, ENFP) & 细节淹没,感到束缚,身体不适 & 怀疑绝症,抑郁退缩,钻牛角尖\tabularnewline
Ne (ISTJ, ISFJ) & 生活未知,计划打乱,快速应变 & 灾难思维,阴谋论,胡乱联系\tabularnewline
Ni (ESTP, ESFP) & 被迫承诺,理论困扰,未来无望 & 宿命论,黑暗妄想,错误解读\tabularnewline
Te (INFP, ISFP) & 价值践踏,自觉无能,强权压迫 & 暴君模式,急功近利,归咎他人\tabularnewline
Ti (ENFJ, ESFJ) & 人际破裂,不被尊重,逻辑混乱 & 冷酷逻辑,过度分析,自我贬低\tabularnewline
Fe (INTP, ISTP) & 被忽视,逻辑无效,情感冲突 & 情绪失控,受害心态,过度情感解读\tabularnewline
Fi (ENTJ, ESTJ) & 权威被挑战,计划失败,感到孤独 & 情感泛滥,被动攻击,甚至抑郁\tabularnewline
\bottomrule
\end{tabular}

\end{table}

\section{认知功能的盲点表现}

\begin{table}[h!]
    \centering
\begin{tabular}{ccc}
\toprule
盲点功能 & 核心特征 & 盲点表现\tabularnewline
\midrule
Fe (INTJ, ISTJ) & 社交迟钝 & 读不懂空气,无意的说真话导致冒犯,视礼仪为虚伪,情感处理滞后\tabularnewline
Fi (ENTP, ESTP) & 道德相对主义 & 不知道自己感受,情感逻辑化,无意间践踏别人价值观\tabularnewline
Se (INTP, INFP) & 现实锻联 & 物理笨拙平地摔,忽视环境,活在云端缺乏行动力\tabularnewline
Si (ENTJ, ENFJ) & 肉体忽视 & 不仅是记性差,透支身体,厌恶怀旧\tabularnewline
Ne (ISTP, ISFP) & 缺乏联想 & 极度字面化,不理解暗示,讨厌猜测,一条路走到黑\tabularnewline
Ni (ESTJ, ESFJ) & 缺乏远见 & 短期主义,只要确定的而不要直觉和愿景,瞎操心\tabularnewline
Te (INFJ, ISFJ) & 逻辑混沌 & 无法通过思考做决定,资源管理困难,维护人际或信仰甚至反智\tabularnewline
Ti (ENFP, ESFP) & 逻辑疏松 & 差不多就行,讨厌理论辩论,逻辑自相矛盾却不自知\tabularnewline
\bottomrule
\end{tabular}

\end{table}



\end{document}
